% !TeX encoding = UTF-8
% !TeX spellcheck = en_US
\documentclass{article}
%%%%%%%%%%%%%%%%%%%%%%%%%%%%%%%%%%%%%%%%%%%%%%%%%%%%%%%%%%%%%%%%%%%%%%%%%%%%%%%%%%%%%%%%%%%%%%%%%%%%%%%%%%%%%%%%%%%%%%%%%%%%%%%%%%%%%%%%%%%%%%%%%%%%%%%%%%%%%%%%%%%%%%%%%%%%%%%%%%%%%%%%%%%%%%%%%%%%%%%%%%%%%%%%%%%%%%%%%%%%%%%%%%%%%%%%%%%%%%%%%%%%%%%%%%%%
\usepackage[a4paper,top=2cm]{geometry}
\usepackage{amssymb,amsmath,amsfonts}
\usepackage[english]{babel}
\usepackage[a4paper]{geometry}
\usepackage{enumitem}
\usepackage{booktabs}
\usepackage{csquotes}
\usepackage{url}
\usepackage{graphicx}
%\parindent0mm
%\parskip1.5ex plus0.5ex minus0.5ex

\begin{document}
	
\title{Computational Macroeconomics}
\author{Midterm Exam}
\date{Summer 2022}
\maketitle


\begin{itemize}
    \item Answer \textbf{all} of the following \textbf{two} exercises in English.
    \item All assignments will be given the same weight in the final grade.
    \item Hand in your solutions before Sunday June, 13 2022 at 3pm.
    \item The solution files should contain your executable (and commented) MATLAB functions and script files
        as well as all additional documentation as \texttt{pdf}, not \texttt{doc} or \texttt{docx}.
    Your \texttt{pdf} files may also include scans or pictures of handwritten notes.
    \item Please e-mail ALL the solution files to \url{willi.mutschler@uni-tuebingen.de}.
    \item I will confirm the receipt of your work also by email (typically within the hour). If not, please resend it to me.		
    \item All students must work on their own, please also give your student ID number.		
    \item It is advised to regularly check Ilias and your emails in case of urgent updates.
    \item If there are any questions, do not hesitate to contact Willi Mutschler.
\end{itemize}
\newpage

\section[An and Schorfheide (2007) Model]{An and Schorfheide (2007) Model\label{ex:AnScho}}
Consider the New Keynesian model of An and Schorfheide (2007, Econometric Reviews).
The model equations are given by 
\begin{align}
	1 &= \beta \mathbb{E}_t\left[\left(\frac{c_{t+1}}{c_t}\right)^{-\tau} \frac{1}{\gamma z_{t+1}} \frac{R_t}{\pi_{t+1}}\right] \label{eq:AS_B1}
  \\
  1 &= \phi \left(\pi_t - \pi\right) \left[\left(1-\frac{1}{2\nu}\right)\pi_t + \frac{\pi}{2\nu}\right] - \phi \beta \mathbb{E}_t \left[\left(\frac{c_{t+1}}{c_t}\right)^{-\tau} \frac{y_{t+1}}{y_t} \left(\pi_{t+1} - \pi \right) \pi_{t+1}\right] + \frac{1}{\nu}\left[1-c_t^{\tau}\right]
  \\
  y_t &= c_t + \frac{g_t-1}{g_t} y_t + \frac{\phi}{2} \left({\pi_t - \pi}\right)^2 y_t
  \\
  R_t &= {R_t^{*}}^{1-\rho_R} R_{t-1}^{\rho_R} e^{\sigma_r\epsilon_{R,t}}
  \\
  \ln(z_t) &= \rho_z \ln(z_{t-1}) + \sigma_z\epsilon_{z,t}
  \\
  \ln(g_t) &= (1-\rho_g)\ln(\bar{g}) + \rho_g \ln(g_{t-1}) + \sigma_g\epsilon_{g,t}
  \\
  y_t^* &= (1-\nu)^{\frac{1}{\tau}} g_t
  \end{align}
where $\epsilon_{R,t}$, $\epsilon_{g,t}$ and $\epsilon_{z,t}$ are iid normally distributed with a standard error of 1.
Moreover, we have the following auxiliary parameter relationships:
\begin{align*}
	  \gamma = 1+\frac{\gamma^{Q}}{100}, \qquad
	  \beta = \frac{1}{1+r^{A}/400}, \qquad
	  \bar{\pi} = 1+\frac{\pi^{A}}{400}, \qquad
	  \phi = \tau \frac{1-\nu}{\nu \bar{\pi}^{2}\kappa}
\end{align*}
and one auxiliary variable $R_t^*$:
\begin{align*}
R_t^* &= R \left(\frac{\pi_t}{\bar{\pi}}\right)^{\psi_1} \left(\frac{y_t}{y_t^*}\right)^{\psi_2}
\end{align*}
Note that $y_t^*$ is an endogenous model variable, whereas $\bar{\pi}$ and $\bar{g}$ are parameters and $R$ denotes the steady-state of $R_t$.

\paragraph{Parametrization}
If not otherwise stated, use the following parameter values for the exercises:
\begin{align*}
	&\tau=2,\quad
	\kappa=0.33,\quad
	\nu=0.1,\quad
	\rho_g=0.95,\quad
	\rho_z=0.9,\quad	
	r^{A}=1,\quad
	\\
	&\pi^{A}=3.2,\quad
	\gamma^{Q}=0.5,\quad
	\bar{g}=1/0.85,\quad
	\sigma_z = \sqrt{0.9},\quad
	\sigma_g=\sqrt{0.36}
  \end{align*}
We will consider two different parametrizations of the monetary policy rule $(\psi_1,\psi_2,\rho_R,\sigma_r)$, one is called \textbf{hawkish}:
\begin{align*}
	&\psi_1=1.5,&& \psi_2=0.125,&& \rho_R = 0.75,&& \sigma_r = \sqrt{0.4}	\\
\end{align*}
and the other one \textbf{dovish}:
\begin{align*}
	&\psi_1=1.043195,&& \psi_2=0.918417,&& \rho_R = 0.792647,&& \sigma_r = \sqrt{0.446783}	
\end{align*}
\newpage
\subsection*{Exercises}

\begin{enumerate}
	\item Provide intuition behind each equation of the model. How are nominal price rigidities modelled?\\\emph{Hint: Have a look at the relevant parts of the underlying paper.}
  \item Write a Dynare mod file for this model.
	\item Derive the steady state of all model variables analytically and include these in a \texttt{steady\_state\_model} block.
   If you struggle to do that, use an \texttt{initval} block.
	\item Calibrate the parameters such that the ratio of steady-state consumption to steady-state output is equal to 90\%.
  For the following exercises, though, revert to the original parametrization.		
	\item Investigate the steady-state properties and theoretical moments of the two different parametrizations (dovish vs. hawkish) 
    by running the \texttt{steady} and \texttt{stoch\_simul(order=1)} commands.\\\emph{Hint: Look at Dynare's output in the command window for the Steady-State values and Theoretical Moments.}
	\item Now compare the impulse-response functions of the two different parametrizations (dovish vs. hawkish) for output and inflation
    to a one standard deviation TFP shock and a one standard deviation monetary policy shock.
  Provide intuition for the impulse response functions.
	\item The determinacy region is defined as the parameter space for which the Blanchard \& Khan (1980, Econometrica) order and rank conditions are full-filled.
  Dynare's sensitivity toolbox provides means to get a graphical representation of this region by drawing randomly from the parameter space and checking the order and rank conditions.
  Add the following code snippet to your mod file:
\begin{verbatim}
%---------------
%specify parameters for which to map sensitivity
estimated_params;
PSI1, uniform_pdf,,, 0,6; %draw uniformly from 0 to 6
PSI2, uniform_pdf,,,-1,6; %draw uniformly from -1 to 6
RHOR, uniform_pdf,,,-1,1; %draw uniformly from -1 to 1
end;
varobs y pie;
dynare_sensitivity(prior_range=0,stab=1,nsam=2000);		
%---------------
\end{verbatim}	
which randomly draws $\psi_1 \in [0,6]$, $\psi_2 \in [-1,6]$ and $\rho_R \in [-1,1]$, while keeping all other parameter values at their calibrated values.
Focus on the the plots and explain these.
What does this imply for the ability of monetary policy to anchor inflation expectations?
\item[Bonus points:] In your opinion, is the monetary policy strategy conducted by the European Central bank hawkish or dovish?
What about the Federal Reserve?
Given the insights you gained in this exercise:
Would it be desirable for policymakers to be rather hawkish or dovish?
\end{enumerate}


\newpage

\section[Brock and Mirman (1972) Model]{Brock and Mirman (1972) Model\label{ex:BrockMirman}}
Consider a simple RBC model with log-utility and full depreciation.
The objective is to maximize
\begin{align*}
\max ~E_t \sum_{j=0}^{\infty} \beta^{j} \log(c_{t+j})
\end{align*}
subject to the law of motion for capital $k_t$ at the end of period $t$
\begin{align}
k_{t} = a_t k_{t-1}^\alpha - c_t \label{eq:BrockMirmanCapital}
\end{align}
with $\beta <1$ denoting the discount factor and $E_t$ is expectation given information at time $t$.
Productivity $a_t$ is the driving force of the economy and evolves according to
\begin{align}
\log{a_{t}} &= \rho_a \log{a_{t-1}}  + \varepsilon_{a,t} \label{eq:BrockMirmanTFP}
\end{align}
where $\rho_a$ denotes the persistence parameter
  and $\varepsilon_{a,t}$ is assumed to be normally distributed with mean zero and variance $\sigma_a$.
Finally, we assume that the transversality condition and 
  the following non-negativity constraints are fullfilled: $k_t \geq0$ and $c_t \geq 0$.

\begin{enumerate}
	\item Show that the first-order condition is given by
	\begin{align}
	1 = \alpha \beta E_t \frac{c_{t}}{c_{t+1}} a_{t+1} k_{t}^{\alpha-1} \label{eq:BrockMirmanEuler}
	\end{align}
	\item Compute the steady-state of the model with pen and paper, in the sense that there is a set of values for the endogenous variables that in equilibrium remain constant over time.
    \item Write MATLAB (\textbf{NOT Dynare}) scripts that:
    \begin{enumerate}
        \item Pre-process the model and save both the static and dynamic model equations as well as Jacobians to script files that can be evaluated for any value of parameters, dynamic endogenous variables and exogenous variables.
        \item Compute the steady-state for the following parametrization: $\alpha=0.3$, $\beta=0.96$, $\rho_a=0.5$, and $\sigma_a=0.2$.
\emph{Hint: You should get the following results:}
\begin{verbatim}
STEADY-STATE RESULTS:
k 		 0.168929
a 		 1
c 		 0.417629
\end{verbatim}
        \item Compute the first-order approximated policy function (i.e. $g_x$ and $g_u$ in the notation used in the lecture) for the following parametrization: $\alpha=0.3$, $\beta=0.96$, $\rho_a=0.5$, and $\sigma_a=0.2$.
\emph{Hint: You should get the following results:}
\begin{verbatim}
POLICY AND TRANSITION FUNCTIONS
                                   k               a               c
Constant                    0.168929        1.000000        0.417629
k(-1)                       0.300000               0        0.741667
a(-1)                       0.084464        0.500000        0.208815
eps_a                       0.168929        1.000000        0.417629
\end{verbatim}
        \item Draw $T=100$ shocks from the standard normal distribution and simulate data for consumption $c_t$ for $t=1,...T$
        by using the first-order approximated policy function.
        \\\emph{Hint: Look at exercise 5 from problem set 1 or exercise 4 from problem set 2 how to simulate with MATLAB.}
    \end{enumerate}
\newpage
 	\item Professor Mutschler taught you the concept of a policy function,
   \begin{align*}
		\begin{pmatrix} c_{t}\\k_{t}\\a_{t} \end{pmatrix}
 		 &= g(a_{t-1},k_{t-1},\varepsilon_{a,t})
 	\end{align*}
   which maps the current state of the economy ($k_{t-1}$ and $a_{t-1}$) and current shocks $\varepsilon_{a,t}$ to current decisions of the agents.
   You remember him stating, that there is no way to derive the function in closed-form, so that's why we use perturbation approximation techniques.
   Willi, a fellow student, disagrees with this assessment and claims that he is able to compute the policy function in closed-form for this model.
   He claims that the policy function is linear in $ a_t k_{t-1}^\alpha$, i.e. it has the form
 	\begin{align}
 	c_{t} &= g^c a_t k_{t-1}^\alpha \label{eq:BrockMirman:gc}\\
 	k_{t} & = g^k a_t k_{t-1}^\alpha \label{eq:BrockMirman:gk}
 	\end{align}
  where $g^c$ and $g^k$ are some scalars which are a function of model parameters $\alpha$ and $\beta$ only.
 	Professor Mutschler is surprised and therefore asks you on the exam to derive the scalar values $g^c$ and $g^k$.
  Do so by inserting the guessed policy functions \eqref{eq:BrockMirman:gc} and \eqref{eq:BrockMirman:gk}
    into the model equations \eqref{eq:BrockMirmanCapital} and \eqref{eq:BrockMirmanEuler}.
 	\item [Bonus points:] Compare simulated data for the endogenous variables as well as impulse response functions of a one-standard-deviation technology shock based on the true solution with the first-order approximated one.
\end{enumerate} 



\end{document}
