\begin{enumerate}

\item First of all, note that in the mod file there is a minus sign in front of the preference shifter \texttt{eps\_z} such that we can answer 1c):
\begin{lstlisting}
[name='preference shifter']
log(z) = RHO_Z*log(z(-1)) - eps_z;
\end{lstlisting}

Consider the following mod file for the stochastic simulations:
\lstinputlisting[style=Matlab-editor,basicstyle=\mlttfamily\scriptsize,title=\lstname]{progs/dynare/nk_irfs_stoch.mod}

Before we dive in, keep in mind that all variables are simultaneously determined;
  any description of the transmission channels can only be a cursory way to provide intuition,
  rather than an accurate characterization of the model's dynamics.
The magic clearly happens through the equations, macroeconomists then try to tell an economic story behind the dynamics (depending on the focus of their analysis).
\\
\paragraph{General remarks for intuition:}
\begin{itemize}
\item Keep in mind that we are in an environment of monopolistic competition,
  constant elasticity demand curves, randomly arriving opportunities to adjust prices,
  and sluggish adjustment of investment due to quadratic costs.
\item When firms set prices, they are concerned about future inflation,
  because there is a chance that they won't be able to adjust prices for several periods.			
\item The inflation process is forward-looking, current inflation is a function of expected future inflation.
More specifically, inflation is the present discounted value of current and future real marginal costs.
\item We have a Cobb-Douglas production function, both labor and capital are substitutes.
\item \emph{Taylor principle}: The monetary feedback parameter in response to inflation deviations is larger than one.
In other words, when inflation rises, the central bank rises the nominal interest rate more than one-to-one.
This guarantees that the real interest rate eventually rises with inflation.
The increase in the real interest rate creates a counter-effect to inflation,
  since a higher real interest rate causes a fall in the output gap and in deviations of the marginal cost from the steady state.
This is the underlying economic principle behind the ability of monetary policy to anchor inflation expectations.
\end{itemize}
Note that in the mod file, we focus on the hat variables which are defined as \textbf{log deviations from steady-state}.
This is useful for interpreting the Impulse Response Functions as Dynare plots these in \textbf{deviation from steady-state}.
So for hat variables this corresponds (at \texttt{order=1}) to \textbf{percentage deviations} from the original level variables from steady state.
  
\paragraph{Technology shock}
The positive unit shock on total factor productivity has on impact a positive effect on output and consumption,
  which then follow a downward sloping path due to the persistence of the technological process.
There is also a boom in investment which has a hump-shaped form.
On the other hand, we see on impact deflation, higher wages, lower hours worked,
  and lower interest rates (both nominal as well as real interest rate),
which then all follow an upward-sloping path back to steady-state.
The hump-shaped response of investment is mirrored in the real rental rate of capital,
  whereas the capital stock adjusts gradually over time.

The intuition is that the boost in productivity increases the marginal productivity of both labor and capital,
  which affects not only the consumption-saving, labor-leisure decision, investment decision,
  but also the price-setting decision of firms.
In more detail, marginal cost of firms falls (as inputs are more productive) and this creates incentives for firms to cut their prices.
The firms that can reset their price are concerned about not being able to reset it again during the future productivity boost period,
  so they lower their prices more than they would under flexible prices,
  implying a larger drop in inflation (compared to the flex-price case) and a drop in real marginal costs.
Likewise, the increase in output is not as large as with flexible prices as the firms that cannot lower their price lower output.
According to the Taylor rule, the central bank reacts to the deflationary pressure by lowering the nominal interest rate more than one-to-one.
In accordance, the real interest rate falls on impact (but not as much as with flexible-prices)
  and then follows an upward sloping path.
This reflects the wish of the households to smooth consumption via the Euler equation,
  as consumption in the future becomes less attractive and households prefer to consume more during the periods of productivity boost.

The effect on wages and hours worked depends on the the calibration of the model;
  particularly, the Cobb-Douglas elasticity, Calvo probability, the utility elasticity parameters and the feedback parameters in the Taylor rule.
In the chosen calibration, we see an increase in wages,  
  a higher Calvo probability might flip the impact effect around.
Similarly, the effect on hours worked, is dependent on the calibration.

The sluggish behavior of capital can be explained by the quadratic adjustment costs in the growth rate of investment,
  which is reflected in the rental rate of capital.
Due to the increase in productivity capital becomes cheaper,
  but investment only sluggishly adjusts until the boom is also reflected in its marginal productivity.
As there is more capital available the real rental rate drops over time,
  and due to the chosen calibration and rather large shock size, this effect is quite persistent.



\paragraph{Preference shifter shock}
The decrease in the preference shifter means that the effective discount factor ($\zeta_t \beta^t$ for $t=0,1,2...$) becomes lower;
  in other words, households are temporarily becoming more patient and prefer to postpone their consumption, therefore output drops on impact.
This decreased demand creates incentives for firms to decrease their price and inflation falls.
Note though, that not all firms can reset their price.
Nevertheless, lower prices imply lower marginal costs, lower wages and lower real rental rate of capital.
According to the Taylor rule, the central bank reacts to the deflationary pressure and decline in output
  by decreasing the nominal interest rate more-than-one-for-one ($\phi_\pi>1$).
In accordance to the Fisherian equation,
  the real interest rate decreases on impact and then follows an upward sloping path.
This reverses the effect on the consumption-savings decision of the households via the Euler equation,
  as consumption in the present becomes again more attractive.
However, the decline in the real interest rate is not sufficient to prevent the overall contraction in economic activity (as in the flex-price model).
Again the effect on hours worked, is in principle ambiguous dependent on the production function and the calibration.
Similarly, the overshooting of output is is related to the adjustment costs that depend on the growth rate of investment
  and the sluggish adjustment of capital.
  
\paragraph{Monetary policy shock}
From the empirical VAR literature we can motivate the \emph{real rate channel of monetary policy}:
  the central bank controls the short-term nominal interest rate and because nominal prices are sticky it has leverage over the ex-ante real interest rate.
Therefore, inducing a demand-side effect by making consumption and investment less attractive.
  
Now let's look at this evidence through the lens of our Baseline New-Keynesian model.
An exogenous tightening of monetary policy, i.e. a positive realization of $\nu_t$,
  indeed replicates all these facts, i.e. output, investment and inflation decline,
  whereas the nominal and real interest rates increase.
In other words, a monetary tightening, in the form of a positive shock to the Taylor rule
  that increases the short-term nominal interest rate translates into an increase in the real interest rate
  as well when nominal prices move sluggishly due to costly or staggered price setting.
This rise in the real interest rate then causes households to cut back on their current consumption spending and investment decisions.
Finally, the decline in output puts downward pressure on inflation, which adjusts only gradually after the shock.

Some notes of caution with this interpretation:
\begin{itemize}
    \item One can show, that for higher values of $\rho_\nu$ the nominal interest rate can decline
      in response to a contractionary monetary policy shock.
    The intuition is the following:
    Assume the initial monetary policy shock increases the nominal and thus the real interest rate.
    The central bank then reacts to lower output and inflation endogenously by lowering the nominal interest rate.
    If the response is strong enough, it overcompensates the initial increase due to the shock.
    The ex-ante real rate, however, always increases, irrespective of the shock persistence in the baseline New Keynesian model,
      as it is inversely related to monetary policy shocks.
    \item A challenge to this \emph{real rate channel of monetary policy} is given by e.g. \textcite{Rupert.Sustek_2019_MechanicsNewKeynesianModels}
      who show that similar to the flexible-price case, inflation is determined by current and expected future monetary policy shocks.
    According to the New-Keynesian Phillips-curve output temporarily drops when inflation temporarily declines.
    The reason is that firms that cannot adjust prices reduce output.
    The real interest rate only reflects the desire and ability of households to keep consumption smooth in face of such temporary changes,
      but it is not the actual driving force of the dynamics.
    Particularly, they show that when introducing capital into the model,
      monetary policy shocks can generate a decline in output and inflation,
      while the reaction of the real interest rate depends on the calibration: it can increase, decline or stay constant.
    \item To sum up, don't look solely at interest rates to see whether monetary policy is expansionary or contractionary.
\end{itemize}

\item Consider the following mod file for deterministic simulations:
\lstinputlisting[style=Matlab-editor,basicstyle=\mlttfamily\scriptsize,title=\lstname]{progs/dynare/nk_irfs_det.mod}

\noindent The helper plotting function is given by \texttt{nk\_irfs\_det\_do\_plots.m:}
\lstinputlisting[style=Matlab-editor,basicstyle=\mlttfamily\scriptsize,title=\lstname]{progs/dynare/nk_irfs_det_do_plots.m}

Overall, the dynamics are very similar to the stochastic case;
  however, often for the state variables the reactions of the perfect foresight solution are stronger in size
  and therefore often do not show the asymptotic behavior back to steady-state,
  but instead overshoot earlier to get back to steady-state sooner.
Note that we did not include a \emph{strong nonlinearity} such as an occasionally binding constraint (zero-lower bound or irreversible investment),
  then the reactions would show much more differences.
Also, the size of the shock matters, the smaller it is the more alike are both simulations.
To sum up, there are slight differences between whether one is more concerned about non-linearities or stochastics.
\end{enumerate}