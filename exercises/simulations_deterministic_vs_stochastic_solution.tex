The key difference between stochastic and deterministic models stems from the role of uncertainty.
  In a deterministic world we have perfect knowledge about all future events including policy actions.
Given some initial data we can derive optimal trajectories leading to a steady state which generates the highest outcome, e.g. total utility flows.
Contrary to that, in a stochastic setting there is always some randomness involved.
That is, the agents do not know if or when a shock will hit the economy.
They do, however, build (mathematical) expectations because the probability distribution of those shocks is known to the agents.
\begin{itemize}
\item     
Consequentially deterministic models can be used if an occurrence of an innovation in the future is completely certain and predictable,
  independent of the duration the shock takes place.
The corresponding so-called \emph{deterministic} simulation provides us with a good impression about the propagation of this shock.
The \emph{unknowns} that we search for are the trajectories of the variables given the model equations,
  not a recursive decision role aka policy function.
We typically use deterministic simulations to study changes in taxes
  or the introduction of a new currency, etc.
Also, when studying transition dynamics from one steady-state to another one deterministic simulations are typically used.
Particularly, when we want to take all non-linearities of the model into account,
  as we do not approximate a policy function or a decision rule.
Now given the nature of perfect foresight,
  there is no uncertainty in the model except when a shock hits on impact (in period 0).
This is important to keep in mind when studying the effects of news shocks or anticipated shocks,
  i.e.\ shocks that are announced on impact, but materialize later.
Only the announcement is a surprise,
  everything else is known with certainty.

\item  
Stochastic simulations rely on the probability distribution of the shocks,
  i.e.\ the exact value and timing of shocks is not known,
  but that they \emph{might} happen and agents do form expectations about that knowing only the distribution of the shocks.
Such models are in a sense more realistic as the future is uncertain
  and agents make probability statements about their decision.
Stochastic simulations are useful to study transmission mechanisms of stochastic shocks (impulse-response analysis),
  how important is the variability of shocks for the variance of the variables (variance decomposition),
  and also to estimate model parameters with data.
The unknowns that we search for are the so-called \emph{policy functions or decision rules} of the agents,
  that describe optimal recursive behavior given the current state of the economy and the current realization of shocks.
There is a downside to use stochastic simulations
  as we need to find the policy function or approximate it numerically.

\end{itemize}