\begin{enumerate}

\item Declaration of variables and parameters:
    \begin{itemize}
    \item \textbf{Endogenous variables}:
    Output $Y_t$,
    consumption $C_t$,
    private investment $I_t$,
    basic government spending $G^B_t$,
    government investment $I_t^G$,
    private capital stock $K_t$, 
    public capital stock $K_t^G$,
    real rental rate of private capital $R^K_{t}$,
    real wage $W_t$,
    income tax rate $\tau_t$,
    fiscal transfers $TR_t$
    and labor $N_t$.
    Note that we have 12 equations, so we should also have 12 endogenous variables.
    \item \textbf{Exogenous variables}:
    Government spending shock $\varepsilon_t^{G^B}$,
    government investment shock $\varepsilon_t^{I^G}$,
    tax rate shock $\varepsilon_t^{\tau}$.
    \item \textbf{Parameters:}
    Discount factor $\beta$,
    utility leisure weight $\theta_L$,
    public capital productivity $\theta_G$,
    private capital productivity $\theta_K$
    labor productivity $\theta_N$,
    capital depreciation rate $\delta_K$,
    persistence parameter government spending rule $\rho_{G^B}$,
    persistence parameter government investment rule $\rho_{I^G}$,
    persistence parameter tax rule $\rho_{\tau}$,
    target value government spending $\bar{G^B}$,
    target value government investment $\bar{I^G}$,
    target value tax rate $\bar{\tau}$,
    \end{itemize}

\item We have 12 parameters to calibrate.
First, we fix the steady-state of six variables using the provided targets:
\begin{align*}
Y &= 1
\\
G^B &= \bar{G}^B = 0.2 Y \tag*{(*)}
\\
I^G &= \bar{I}^G = 0.02 Y \tag*{(*)}
\\
TR &= 0
\\
W &= 2
\\
N &= \bar{N} = 1/3
\end{align*}
Second, from the labor demand equation in steady-state we have:
\begin{align*}
\theta_N = \frac{W N}{Y} = 2/3 \tag*{(*)}
\end{align*}
Constant returns to scale over privately provided inputs in the production function implies
\begin{align*}
\theta_K = 1-\theta_N = 1/3 \tag*{(*)}
\end{align*}
Third, public capital productivity $\theta_G$ should be lower than $\theta_K$, so for example:
\begin{align*}
\theta_G = 0.3 \cdot \theta_K = 0.1 \tag*{(*)}
\end{align*}
Fourth, capital depreciation rate is supposed to be 2.5\%:
\begin{align*}
\delta_K = 0.025 \tag*{(*)}
\end{align*}
Then evaluating the public capital accumulation equation, the production function, private capital accumulation equation, capital demand, market clearing and fiscal budget in steady-state imply:
\begin{align*}
K^G &= \frac{I^G}{\delta_K} = 0.8
\\
K &= \left(\frac{Y}{(K^G)^{\theta_G} N^{\theta_N}}\right)^{\frac{1}{\theta_K}} \approx 9.6231
\\
I &= \delta_K K \approx 0.2406
\\
C &= Y - I - G^B - I^G \approx 0.5394
\\
R^K &= \theta_K \frac{Y}{K} \approx 0.0364
\\
\tau &= \bar{\tau} = \frac{G^B + I^G + TR}{W N + R^K K} \approx 0.2164 \tag*{(*)}
\end{align*}
From the savings decision in steady-state we get
\begin{align*}
\beta = \frac{1}{1-\delta + (1-\tau)R^K} \approx 0.9965 \tag*{(*)}
\end{align*}
Now we can compute the labor utility weight parameter from the labor supply decision:
\begin{align*}
\theta_L = (1-\tau) W \frac{1-N}{C} \approx 1.9369 \tag*{(*)}
\end{align*}
Lastly, the persistence parameters are supposed to be equal to $0.75$:
\begin{align*}
\rho_{G^B} &= 0.75 \tag*{(*)}
\\
\rho_{I^G} &= 0.75 \tag*{(*)}
\\
\rho_{\tau} &= 0.75 \tag*{(*)}
\end{align*}
Counting the $*$, we have calibrated the 12 model parameters using the steady-state relationships and the provided targets.

\item The mod file might look like this:
\lstinputlisting[style=Matlab-editor,basicstyle=\mlttfamily\scriptsize,title=\lstname]{progs/dynare/Baxter_King_steady.mod}

\item 
\begin{enumerate}    
    \item A balanced growth path refers to a situation in which all model variables grow at some constant rate.
    This is motivated by certain stylized facts (sometimes referred to as Kaldor facts or great ratios):
    \begin{itemize}
        \item The shares of national income received by labor and capital are roughly constant over long periods of time
        \item The rate of growth of the capital stock is roughly constant over long periods of time
        \item The rate of growth of output per worker is roughly constant over long periods of time
        \item The capital/output ratio is roughly constant over long periods of time
        \item The rate of return on investment is roughly constant over long periods of time
        \item The real wage grows over time
    \end{itemize}
    \item \textcite{King.Plosser.Rebelo_1988_ProductionGrowthBusiness} show that Harrod neutral technical progress
    and certain preferences (like the log-utility case we have specified here) are needed in the RBC model for the existence of a Balanced Growth Path.
    Moreover, the production function needs to exhibit constant returns to scale, i.e.\ $\theta_N+\theta_K+\theta_G=1$.
    \item First, note that time endowment is fixed according to $L_t+N_t=1$, so both $L_t$ and $N_t$ have a balanced growth rate of 0.
    We will now derive the balanced growth path (BGP).
    First, we postulate that on the BGP output, private and public capital grow at constant rates $\gamma_Y$, $\gamma_K$, and $\gamma_{K^G}$.
    Let's revisit the production function \eqref{eq:BaxterKing:ProductionFunctionGrowth} and divide both sides by $X_t$
    \begin{align*}
    \frac{Y_t}{X_t} = \left(\frac{X_t N_t}{X_t}\right)^{\theta_N} \left(\frac{K_{t-1}}{X_t}\right)^{\theta_K} \left(\frac{K^G_{t-1}}{X_t}\right)^{\theta_G}
    \end{align*}
    where we make use of constant returns to scale $\theta_N+\theta_K+\theta_G=1$.    
    As $N_t$ does not grow over time, this equation implies that if $Y_t$, $K_{t-1}$ and $K_{t-1}^G$ are growing at the same rate as $X_t$,
      then they are on a BGP.
    Note that the constant returns to scale assumption is crucial here.
    So let's postulate $\gamma_Y=\gamma_K=\gamma_{K^G}=\gamma_X$ and check the other equations if there is a contradiction to such a BGP. 
    Further, let's denote with lowercase letters detrended variables:
    \begin{align*}
    y_t \equiv \frac{Y_t}{X_t}, \qquad k_{t-1} \equiv \frac{K_{t-1}}{X_t}, \qquad k^G_{t-1} \equiv \frac{K^G_{t-1}}{X_t}
    \end{align*}
    such that the stationarized production function is
    \begin{align*}
    y_t = N_t^{\theta_N} (k_{t-1})^{\theta_K} (k^G_{t-1})^{\theta_G}
    \end{align*}
    Let's stationarize the labor demand equation \eqref{eq:BaxterKing:LaborDemand}:
    \begin{align*}
    \underbrace{\frac{W_t}{X_t}}_{w_t} N_t = \theta_N \frac{Y_t}{X_t} \Leftrightarrow w_t N_t = \theta_N y_t
    \end{align*}
    The right-hand side is constant and does not grow.
    Accordingly, the real wage grows over time (see our stylized facts), and on a BGP it needs to actually grow with $\gamma_X$,
      because otherwise the left-hand side, i.e.\ the ratio $W_t/X_t$, wouldn't be constant over time.
    Similarly, let's stationarize the capital demand equation
    \begin{align*}
    R^K_t \frac{K_{t-1}}{X_t} = \theta_K \frac{Y_t}{X_t} \Leftrightarrow R^K_t k_{t-1} = \theta_K y_t
    \end{align*}    
    As $k_{t-1}$ and $y_t$ are already detrended, there is no need to further detrend $R^K_t$,
      i.e.\ on a BGP the rate of return is constant (see our stylized facts).
    So far we basically replaced the upper-case letters with lower-case (detrended) letters,
      but the equations remain the same.
    We can do the same for the market clearing condition as well as the fiscal budget:
    \begin{align*}
    y_t &= c_t + i_t + g_t^B + i_t^G
    \\
    g_t^B + i_t^G &= \tau_t(W_t N_t + R^K_t k_{t-1}) - tr_t
    \end{align*}
    The fiscal rules need to be changed to correspond to detrended variables (in a sense these are just ad-hoc rules and not based on optimality conditions):
    \begin{align*}
    g_t^B-\bar{g}^B &= \rho_{G^B} \left(g_{t-1}^B - \bar{g}^B\right) +  \varepsilon_t^{g^B}
    \\
    i_t^G-\bar{i}^G &= \rho_{I^G} \left(i_{t-1}^G - \bar{i}^G\right) +  \varepsilon_t^{i^G}
    \\
    \log\left(\frac{\tau_t}{\bar{\tau}}\right) &= \rho_\tau \log\left(\frac{\tau_{t-1}}{\bar{\tau}}\right) +  \varepsilon_t^\tau
    \end{align*}
    The consumption-leisure choice is given in detrended form:
    \begin{align*}
    (1-\tau_t) w_t = \theta_L \frac{c_t}{1-N_t}
    \end{align*}
    Finally, we turn to the equations that do change, i.e.\ dividing the capital accumulation equations by $X_t$ yields:
    \begin{align*}
    \frac{K_t}{X_{t+1}}\frac{X_{t+1}}{X_t} = (1-\delta_K)\frac{K_{t-1}}{X_t} + \frac{I_t}{X_t} \Leftrightarrow \gamma_X k_{t} =  (1-\delta_K) k_{t-1} + i_t
    \\
    \frac{K^G_t}{X_{t+1}}\frac{X_{t+1}}{X_t} = (1-\delta_K)\frac{K^G_{t-1}}{X_t} + \frac{I^G_t}{X_t} \Leftrightarrow \gamma_X k^G_{t} =  (1-\delta_K) k^G_{t-1} + i^G_t
    \end{align*}
    Similarly, the detrended Euler equation changes to:
    \begin{align*}
    \frac{ X_{t+1} \frac{C_{t+1}}{X_{t+1}} } {X_t \frac{C_{t}}{X_{t}}} = \gamma_X \frac{c_{t+1}}{c_t} = \beta E_t \left[ 1-\delta_K+(1-\tau_{t+1}r_{K,t+1}) \right]
    \end{align*}
    
\item In the codes, we need to impose constant returns to scale in the production function and adjust the capital accumulation equations and the capital Euler equation.
This also requires some changes in the computation\texttt{steady\_state\_model} block.
The mod file might look like this:
\lstinputlisting[style=Matlab-editor,basicstyle=\mlttfamily\scriptsize,title=\lstname]{progs/dynare/Baxter_King_steady_growth.mod}
Note that we normalized steady-state output higher than 1, because otherwise steady-state consumption would become negative.
    
\end{enumerate}


\end{enumerate}