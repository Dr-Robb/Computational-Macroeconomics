\section[Identification and Sensitivity Analysis]{Identification and Sensitivity Analysis\label{ex:AnScho2007_identif_bk}}
Consider the New Keynesian model of \textcite{An.Schorfheide_2007_BayesianAnalysisDSGE}.
The model equations are given by 
\begin{align}
1 &= \beta \mathbb{E}_t\left[\left(\frac{c_{t+1}}{c_t}\right)^{-\tau} \frac{1}{\gamma z_{t+1}} \frac{R_t}{\pi_{t+1}}\right] \label{eq:AS_Euler}
\\
1 &= \phi \left(\pi_t - \pi\right) \left[\left(1-\frac{1}{2\nu}\right)\pi_t + \frac{\pi}{2\nu}\right] - \phi \beta \mathbb{E}_t \left[\left(\frac{c_{t+1}}{c_t}\right)^{-\tau} \frac{y_{t+1}}{y_t} \left(\pi_{t+1} - \pi \right) \pi_{t+1}\right] + \frac{1}{\nu}\left[1-c_t^{\tau}\right] \label{eq:AS_PC}
\\
y_t &= c_t + \frac{g_t-1}{g_t} y_t + \frac{\phi}{2} \left({\pi_t - \pi}\right)^2 y_t \label{eq:AS_Market}
\\
R_t &= {R_t^{*}}^{1-\rho_R} R_{t-1}^{\rho_R} e^{\epsilon_{R,t}} \label{eq:AS_Taylor1}
\\
R_t^* &= R \left(\frac{\pi_t}{\bar{\pi}}\right)^{\psi_1} \left(\frac{y_t}{y_t^*}\right)^{\psi_2} \label{eq:AS_Taylor2}
\\
\ln(z_t) &= \rho_z \ln(z_{t-1}) + \epsilon_{z,t} \label{eq:AS_tfpgrowth}
\\
\ln(g_t) &= (1-\rho_g)\ln\left(\bar{g}\right) + \rho_g \ln(g_{t-1}) + \epsilon_{g,t} \label{eq:AS_g}
\\
y_t^* &= (1-\nu)^{\frac{1}{\tau}} g_t \label{eq:AS_flexy}
\end{align}
where $\epsilon_{R,t}$, $\epsilon_{g,t}$ and $\epsilon_{z,t}$ are iid normally distributed with a mean of zeros and standard errors equal to $\sigma_r$, $\sigma_g$, $\sigma_z$, respectively.
Moreover, we have the following auxiliary parameter relationships:
\begin{align*}
\gamma = 1+\frac{\gamma^{Q}}{100}, \qquad
\beta = \frac{1}{1+r^{A}/400}, \qquad
\bar{\pi} = 1+\frac{\pi^{A}}{400}, \qquad
\phi = \tau \frac{1-\nu}{\nu \bar{\pi}^{2}\kappa}
\end{align*}
Note that $y_t^*$ is an endogenous model variable, whereas $\bar{\pi}$ and $\bar{g}$ are parameters and $R$ denotes the steady-state of $R_t$.

\paragraph{Parametrization}
If not otherwise stated, use the following parameter values for the exercises:
\begin{align*}
&\tau=2,\quad
\kappa=0.33,\quad
\nu=0.1,\quad
\rho_g=0.95,\quad
\rho_z=0.9,\quad	
r^{A}=1,\quad
\\
&\pi^{A}=3.2,\quad
\gamma^{Q}=0.5,\quad
\bar{g}=1/0.85,\quad
\sigma_z = \sqrt{0.9},\quad
\sigma_g=\sqrt{0.36}
  \end{align*}
We will consider two different parametrizations of the monetary policy rule $(\psi_1,\psi_2,\rho_R,\sigma_r)$, one is called \textbf{hawkish}:
\begin{align*}
&\psi_1=1.5,&& \psi_2=0.125,&& \rho_R = 0.75,&& \sigma_r = \sqrt{0.4}	\\
\end{align*}
and the other one \textbf{dovish}:
\begin{align*}
&\psi_1=1.043195,&& \psi_2=0.918417,&& \rho_R = 0.792647,&& \sigma_r = \sqrt{0.446783}	
\end{align*}

\newpage

\subsection*{Exercises}

\begin{enumerate}

\item Provide intuition behind each equation of the model.
How are nominal price rigidities modelled?\\\emph{Hint: Have a look at the relevant parts of the underlying paper.}

\item Write a Dynare mod file for this model. Derive the steady state of all model variables analytically and include these in a \texttt{steady\_state\_model} block.

\item Explain how theoretical moments are computed based on the first-order policy function.

\item Investigate the steady-state properties, theoretical moments, and impulse response functions of the two different parametrizations (dovish vs. hawkish) 
  by running the \texttt{stoch\_simul(order=1)} command.
\\
\emph{Hint: Look at Dynare's output in the command window for the Steady-State values and Theoretical Moments.}

\item The determinacy region is defined as the parameter space for which the Blanchard \& Khan order and rank conditions are full-filled.
Dynare's sensitivity toolbox provides means to get a graphical representation of this region
  by drawing randomly from the parameter space and checking the order and rank conditions.
Add the following code snippet to your mod file:
\begin{verbatim}
%---------------
%specify parameters for which to map sensitivity
estimated_params;
PSI1, uniform_pdf,,, 0,6; %draw uniformly from 0 to 6
PSI2, uniform_pdf,,,-1,6; %draw uniformly from -1 to 6
RHOR, uniform_pdf,,,-1,1; %draw uniformly from -1 to 1
end;
varobs y pie R;
dynare_sensitivity(prior_range=0,stab=1,nsam=2000);		
%---------------
\end{verbatim}	
  which randomly draws $\psi_1 \in [0,6]$, $\psi_2 \in [-1,6]$ and $\rho_R \in [-1,1]$,
  while keeping all other parameter values at their calibrated values.
Focus on the the plots and explain these.
What does this imply for the ability of monetary policy to anchor inflation expectations?
\end{enumerate}

\paragraph{Readings}
\begin{itemize}
	\item \textcite{An.Schorfheide_2007_BayesianAnalysisDSGE}
	\item \textcite{Ivashchenko.Mutschler_2020_EffectObservablesFunctionala}
\end{itemize}

\begin{solution}\textbf{Solution to \nameref{ex:AnScho2007_identif_bk}}
\ifDisplaySolutions
\begin{enumerate}

\item Interpretation of model equations:
\begin{itemize}
    \item Equation \eqref{eq:AS_Euler} is the Euler equation determining optimal consumption and saving decision.
    \item Equation \eqref{eq:AS_PC} is the nonlinear New Keynesian Phillips curve,
      which determines optimal price setting of the firms subject to Rotemberg price adjustment costs.
    \item Equation \eqref{eq:AS_Market} is the market clearing condition;
      note that the price adjustment costs are quadratic and valued in terms of output goods.
    \item Equations \eqref{eq:AS_Taylor1} and \eqref{eq:AS_Taylor2} form the Taylor rule,
      i.e.\ monetary policy reacts to deviations of both inflation from its target
      as well as the output gap, which is defined as deviation of output from its flex-price version.
    \item Equation \eqref{eq:AS_tfpgrowth} is the law of motion of the growth rate of productivity.
    \item Equation \eqref{eq:AS_g} is the law of motion for the government spending process.
    \item Equation \eqref{eq:AS_flexy} is the output value which would occur when there are no price rigidties ($\phi=0$).
\end{itemize}

\item The mod file might look like this:
\lstinputlisting[style=Matlab-editor,basicstyle=\mlttfamily\scriptsize,title=\lstname]{progs/dynare/an_schorfheide_identif_bk.mod}

\item The first-order approximated policy function is given by:
\begin{align*}
    y_t = \bar{y} + g_y (y_{t-1} - \bar{y}) + g_u u_t
\end{align*}
where $g_y$ is a $n \times n$ matrix.
Note that Dynare's \texttt{oo\_.dr.ghx} matrix has only columns for state variables,
  so $g_y$ is basically equal to: $g_y = \begin{bmatrix} 0_{n \times n^{static}} & ghx & 0_{n \times n^{fwrd}} \end{bmatrix}$.

Due to the Blanchard-Khan conditions this is a stationary process,
  such that $E[y_t] = E[y_s] = \mu_y$ and $\Sigma_{j} \equiv COV[y_t,y_{t-j}] = COV[y_{s}, y_{s-j}]$ for all $s,t,j$.
The first moment is equal to the steady-state:
\begin{align*}
&E [y_t] = \bar{y} + g_y (E [y_{t-1}] - \bar{y}) + g_u E [u_t]
\\
&\Leftrightarrow \mu_y = \bar{y} + g_y(\mu_y - \bar{y}) + g_u \cdot 0
\\
&\Leftrightarrow \mu_y = \bar{y}
\end{align*}
Let's denote $\hat{y}_t = y_t - \bar{y}$, then the variance of $y_t$ is given by:
\begin{align*}
&\Sigma_0 = E[\hat{y}_t \hat{y}_t'] = g_y E[\hat{y}_{t-1} \hat{y}_{t-1}'] g_y' + g_u E[u_t u_t'] g_u'
\\
&\Leftrightarrow \Sigma_0 = g_y \Sigma_0 g_y' + g_u \Sigma_u g_u'
\end{align*}
This is a Lyapunov equation which you can either solve with the Kronecker product or an efficient algorithm (e.g.\ doubling).
Once we have $\Sigma_0$, the autocovariances are computed by $\Sigma_j = g_y \Sigma_{j-1}$.

\item Here is the mod file for the hawkish calibration:
\lstinputlisting[style=Matlab-editor,basicstyle=\mlttfamily\scriptsize,title=\lstname]{progs/dynare/an_schorfheide_identif_hawkish.mod}

Here is the mod file for the dovish calibration:
\lstinputlisting[style=Matlab-editor,basicstyle=\mlttfamily\scriptsize,title=\lstname]{progs/dynare/an_schorfheide_identif_dovish.mod}

Note that the steady-state and theoretical moments are (numerically) equal to each other.
Moreover, the IRFs look exactly the same (and they actually are at least numerically).

This is a manifestation of an inherent \emph{identification problem} in the model:
  different structures yield the same moments and impulse responses;
  they cannot be distinguished from each other.
This is quite a severe problem in this specific model,
  because according to the model there is no distinction between a hawkish or a dovish monetary policy,
  which is obviously counter-factual.
Note that this identification problem is a property of the model
  and needs to be dealt with, i.e.\ changing your model slightly by adding features to the model
  or a different specification for the monetary policy rule
  solves the identification problem, see \textcite{Ivashchenko.Mutschler_2020_EffectObservablesFunctionala} for more details.

\item Here is the mod file that does the sensitivity analysis of the Blanchard-Khan conditions:
\lstinputlisting[style=Matlab-editor,basicstyle=\mlttfamily\scriptsize,title=\lstname]{progs/dynare/an_schorfheide_sensitivity_bk.mod}
Note that in the figures blue dots mark parameter values that imply \emph{good behavior},
  whereas red dots imply \emph{bad behavior}.
In our exercise \emph{good behavior} corresponds to a unique and stable solution,
  according to the Blanchard and Khan order and rank conditions.
If you study the graphs in detail, one sees that particularly the feedback parameter on inflation deviations drives the behavior.
In almost all cases $\psi_1>1$ implies that the model has a unique and stable solutions.
This is known as the Taylor Principle, i.e.\ a guideline or rule of thumb that suggests that a central bank should raise nominal interest rates
  by more than one-for-one in response to an increase in inflation.
The underlying mechanism is that the central bank controls not only the short-term nominal interest rate,
  but because nominal prices are sticky it has leverage over the ex-ante real interest rate.
Therefore, by raising nominal interest rates sufficiently, the real interest rate will increase as well inducing a demand-side effect by making consumption and investment today less attractive.
If a central bank were to raise the nominal interest rate by less than the increase in inflation,
  then the real interest rate (the nominal rate minus inflation) could actually decrease, which could stimulate demand and potentially lead to even more inflation.
But if the central bank raises the nominal rate by more than the increase in inflation, then the real interest rate increases, which tends to reduce demand and curb inflation.

The Taylor Principle has significant implications for the ability of monetary policy to anchor inflation expectations.
By following a policy rule that raises interest rates more than one-for-one with inflation,
  central banks can help to stabilize inflation and anchor inflation expectations.
Here's why: If the public expects the central bank to adhere to the Taylor Principle,
  they'll anticipate that any rise in inflation will be met with an even greater rise in nominal interest rates,
  leading to an increase in the real interest rate.
This should in turn suppress demand and curb further inflation.
If people expect that the central bank will keep inflation under control in this way, their inflation expectations will be stable, which is beneficial for the economy.
Stable inflation expectations can reduce the volatility of inflation and output, leading to smoother economic cycles.
However, it's important for the central bank to have a credible commitment to such a policy.
If the public doubts the central bank's willingness or ability to respond aggressively to inflation, they might expect higher future inflation.
This could be self-fulfilling, as workers demand higher wages and firms raise prices in anticipation of the higher inflation.
In other words, the Taylor Principle not only provides a guide for how to adjust interest rates in response to inflation,
  but it also offers a way to manage the public's expectations about future inflation.
This is a crucial aspect of modern monetary policy, as expectations about future inflation can influence current behavior and can therefore become self-fulfilling.
\end{enumerate}
\fi
\newpage
\end{solution}