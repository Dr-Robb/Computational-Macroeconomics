\section[Case Study Eichenbaum, Rebelo and Trabandt (2022, JEDC): Epidemics in the New Keynesian model]{Case Study Eichenbaum, Rebelo and Trabandt (2022, JEDC): Epidemics in the New Keynesian model\label{ex:CaseStudy.Eichenbaum.Rebelo.Trabandt.2022}}

\begin{enumerate}
\item Read the paper by \textcite{Eichenbaum.Rebelo.Trabandt_2022_EpidemicsNewKeynesian}
  and focus particularly on Section 3 (The model economy), Section 4 (The impact of an epidemic), and the Appendix A (Equilibrium equations).

\item Consider the Dynare mod file given in Appendix \ref{app:ert_model_sticky}, which is a stripped down version of the sticky-price New Keynesian SIR model as outlined in Appendix A of the paper.
Compute the steady-state using an \texttt{initval} block.

\item Note that the mod file contains a \texttt{predetermined\_variables} command.
Read about this in the manual of Dynare \parencite{Adjemian.Bastani.Juillard.EtAl_2022_DynareReferenceManual}
  and explain why this is relevant in the current model.

\item Note that the Taylor rule, in equation 31), exhibits differences between the mod file and the paper.
Specifically, in the paper, the output gap is defined as the discrepancy between output and the corresponding output level under flexible prices ($\xi=0$),
  whereas in the mod file, it is defined as deviations from the steady-state.
Generate a new mod file that incorporates equations for both the sticky-price and flexible-price economies,
  with the two models connected through the Taylor rule in the sticky model.
Additionally, you need to recalculate the pre-infectious steady-state for the combined economies.
\emph{Hint: Don't forget to set the macro directive by including:\\ \texttt{\@\#define FLEX\_PRICE\_OUTPUT\_GAP = 1}}

\item Read section \texttt{4.7. Initial and terminal conditions} in the manual of Dynare \parencite{Adjemian.Bastani.Juillard.EtAl_2022_DynareReferenceManual}.
Add a \texttt{histval} block that implements the impact of an epidemic in the model,
  i.e. in period $t=0$ change infected to $i_0=\varepsilon$ and susceptibles to $s_0=1-\varepsilon$,
  while all other variables are set to their pre-infection steady-state.
Run \texttt{perfect\_foresight\_setup} and have a look how Dynare initializes \texttt{M\_.\_endo\_histval},
  \texttt{oo\_.\_endo\_simul} and \texttt{oo.\_exo\_simul} for the Newton algorithm.

\item Notice that the \texttt{model} block has a \texttt{differentiate\_forward\_vars} option.
Look into the manual of Dynare \parencite{Adjemian.Bastani.Juillard.EtAl_2022_DynareReferenceManual} what this option does.
Explain why it is necessary in this model when we want to simulate the impact of an epidemic?

\item What is \emph{homotopy} in the context of perfect foresight simulations?
Illustrate this by setting
\begin{align*}
\pi_1 &= 2.568436063602094 \times 10^{-7}\\
\pi_2 &= 1.593556989906377  \times 10^{-4}\\
\pi_3 &= 0.499739472034583
\end{align*}
  and running a perfect foresight solver using the above \texttt{histval} block.

\newpage

\item Simulate the economy when the epidemic is a shock to consumption demand only,
  i.e.\ replicate Figures 5 and 6.
To this end, calibrate the probabilities according to:

{\footnotesize
\begin{lstlisting}[style=Matlab-editor,basicstyle=\mlttfamily\scriptsize]
% calibation targets for shares of pi-terms in T-function in SIR model
pi3_shr_target = 2/3;                  % share of T_0 jump due general infections
pi1_shr_target = (1-pi3_shr_target);   % share of T_0 jump due to consumption-based infections
pi2_shr_target = 0;                    % share of T_0 jump due to work-based infections
[pi1_final,pi2_final,pi3_final] = ert_model_go_calibrate_pi(250,varepsilon,pir,pid,pi1_shr_target,pi2_shr_target,RplusD_target,c_ss,n_ss);
\end{lstlisting}
}

\item Simulate the economy when the epidemic is a shock to labor supply only,
  i.e.\ replicate Figures 7 and 8.
To this end, calibrate the probabilities according to:

{\footnotesize
\begin{lstlisting}[style=Matlab-editor,basicstyle=\mlttfamily\scriptsize]
pi3_shr_target = 2/3;                  % share of T_0 jump due general infections
pi1_shr_target = 0;                    % share of T_0 jump due to consumption-based infections
pi2_shr_target = (1-pi3_shr_target);   % share of T_0 jump due to work-based infections
[pi1_final,pi2_final,pi3_final] = ert_model_go_calibrate_pi(250,varepsilon,pir,pid,pi1_shr_target,pi2_shr_target,RplusD_target,c_ss,n_ss);
\end{lstlisting}
}

\item Simulate the economy when the epidemic is a shock to both consumption demand and labor supply,
  i.e.\ replicate Figures 9 and 10.
To this end, calibrate the probabilities according to:

{\footnotesize
\begin{lstlisting}[style=Matlab-editor,basicstyle=\mlttfamily\scriptsize]
pi3_shr_target = 2/3;                  % share of T_0 jump due general infections
pi1_shr_target = (1-pi3_shr_target)/2; % share of T_0 jump due to consumption-based infections
pi2_shr_target = (1-pi3_shr_target)/2; % share of T_0 jump due to work-based infections
[pi1_final,pi2_final,pi3_final] = ert_model_go_calibrate_pi(250,varepsilon,pir,pid,pi1_shr_target,pi2_shr_target,RplusD_target,c_ss,n_ss);
\end{lstlisting}
}
    
\end{enumerate}


\paragraph{Hints}

\begin{itemize}

\item Appendix \ref{app:ert_model_go_calibrate_pi} contains a helper function to calibrate the probabilities $\pi_1$, $\pi_2$ and $\pi_3$.
    
\item Appendix \ref{app:ert_model_plot_agg_results} contains a helper function to plot figures 5, 7, and 9.

\item Appendix \ref{app:ert_model_plot_by_type_results} contains a helper function to plot figures 6, 8, and 10.

\item It might help to know that you can have multiple \texttt{model} and \texttt{var} blocks in the same mod file.

\item Dynare's \texttt{\@\#include "SOMEMODILE"} directive might be useful.

\item \texttt{set\_param\_value('pi3',pi3\_final)} is a useful function to update parameters when you loop over parameters.

\item If your solution does not converge, try homotopy for $\pi_3$, e.g. start with $\pi_3/3$.

\item Set the \texttt{periods} option to 500 and the \texttt{maxit} option to 100.

\end{itemize}


\begin{solution}\textbf{Solution to \nameref{ex:CaseStudy.Eichenbaum.Rebelo.Trabandt.2022}}
\ifDisplaySolutions
\begin{itemize}

    \item[2.] Let's use Dynare's \texttt{@\#include "ert\_model\_sticky.mod"} directive to read in the sticky-price mod file into another mod file that just computes the steady-state.
Note that during calibration all values for the pre-infection steady-state are already computed (with a subindex \texttt{\_ss}),
  so we can simply put that into an \texttt{initval} block:
\lstinputlisting[style=Matlab-editor,basicstyle=\mlttfamily\scriptsize,title=\lstname]{progs/replications/Eichenbaum_Rebelo_Trabandt_2022/ert_model_sticky_steady.mod}

\item[3.] In the paper the state variables capital, infected, susceptible, recovered, deceased and population
  have a "beginning-of-period" timing convention,
  whereas Dynare requires these variables to be "end-of-period".
One could either lag all occurrences manually in the model equations,
  or use \texttt{"predetermined\_variables"} such that Dynare's preprocessor does this transformation of timing.
This is personal preference, but using \texttt{"predetermined\_variables"} makes the equations appear more in line with the paper.

\item[4.] This is basically a copy \& paste exercise combined with the \texttt{@\#include} directive.
First, let's create a mod file for the flex-price economy called \texttt{"ert\_model\_flex.inc"}
\begin{itemize}
    \item Copy and paste the variable declarations and model equations of \texttt{ert\_model\_sticky.mod} into the new mod file.
    \item Add a \texttt{\_flex} to the variables to indicate that this is the same variable,
      but in the flex-price model.
    Similarly, adjust all appearances of the variable in the model equations.
    Hint: You might want to use MATLAB's \emph{Find \& Replace} feature for this.
    \item Rename the parameter $\xi$ in the flex-price model to $\xi^f$ and set it equal to 0.
    Do so by either declaring a new parameter calibrated to 0 or using a \emph{model local variable.}
    \item Re-compute manually the steady-states of the auxiliary variables in the recursive pricing equations $K^{f,flex}$ and $F^{flex}$.
\end{itemize}
The inc file might look like this:
\lstinputlisting[style=Matlab-editor,basicstyle=\mlttfamily\scriptsize,title=\lstname]{progs/replications/Eichenbaum_Rebelo_Trabandt_2022/ert_model_flex.inc}
As a mod file can contain several declarations of variables (\texttt{var} blocks) or model equations (\texttt{model} blocks),
  we can conveniently use \texttt{@\#include} to put both economies into the same mod file.
So let's create a new mod file called \texttt{"ert\_model\_initval.mod"}
  and use \texttt{@\#include "ert\_model\_sticky.mod"} to read in the sticky-price mod file
  and similarly \texttt{@\#include "ert\_model\_flex.inc"} directive to read in the flex-price mod file.
Then compute the pre-infection steady-state by using an \texttt{initval} block.
Note that we need to set the macro directive\\\texttt{@\#define FLEX\_PRICE\_OUTPUT\_GAP = 1}\\
  prior to including the flex-price model so the correct Taylor rule is chosen.
\lstinputlisting[style=Matlab-editor,basicstyle=\mlttfamily\scriptsize,title=\lstname]{progs/replications/Eichenbaum_Rebelo_Trabandt_2022/ert_model_initval.mod}

\item[5.] The key sentence in the manual interesting to our application is:
\begin{quote}
\emph{Another one is to study how an economy, starting from arbitrary initial conditions at time 0 converges towards equilibrium.
In this case models, the command histval permits to specify different historical initial values for variables with lags for the periods before the beginning of the simulation.
Due to the design of Dynare, in this case initval is used to specify the terminal conditions.}
\end{quote}
For us the arbitrary initial condition is that infections break out at time 0
  and we want to study how the economy converges towards equilibrium (which is not the pre-infection steady-state, see next exercise).
\lstinputlisting[style=Matlab-editor,basicstyle=\mlttfamily\scriptsize,title=\lstname]{progs/replications/Eichenbaum_Rebelo_Trabandt_2022/ert_model_histval.mod}

\item[6.] The Newton algorithm requires the evaluation of the forward-looking variables at the terminal steady-state.
But the terminal steady-state is an endogenous function of the epidemic dynamics;
  hence, it is not known a priori.
Typically, this makes the perfect foresight solution algorithm inaccurate and often the solution cannot be found.
But notice that the problem is just that we need to evaluate ONLY the forward-looking variables.
So, one way out is to replace all forward-looking variables in the model by auxiliary variables
  that have a known terminal steady-state, such that the solution regains numerical accuracy.
Dynare's \texttt{differentiate\_forward\_vars} option implements the following transformation automatically:
Replace all (or a subset of) variables dated t+1, say $X_{t+1}$,
  by the sum of the current level and expected variation, $X_t + dX_{t+1}$,
  and introduce an auxiliary equation for the new variable $dX_t=X_t-X_{t-1}$.
This transformation is done automatically by Dynare's preprocessor,
  which is a very good thing because doing such transformations manually is extremely error-prone.

\item[7.] Newton-type algorithms are very powerful if we are close to the actual solution
  as it belongs to the family of gradient-type optimizers.
Whenever we want to study \textbf{large} deviations from an initial condition,
  Newton-type algorithms might have a hard time to find the solution.
The idea of homotopy (also called divide-and-conquer) is therefore to subdivide the problem of finding the solution into smaller problems.
So for example, if you have a large exogenous shock on impact of size $1$,
  the idea is to first find a solution for a smaller shock size, say $0.1$,
  and then use that as an initial condition to find a solution for a shock size of $0.2$, $0.3$,..., and finally $1$.
This is the default of Dynare.
\begin{quote}
\emph{Whenever the maximum number of iterations is reached by the Newton algorithm,
  the perfect foresight solver uses a homotopy technique if it cannot solve the problem.
Concretely, it divides the problem into smaller steps by diminishing the size of shocks
  and increasing them progressively until the problem converges.}
\end{quote}
But in our case we don't have a shock,
  but jump from no infections to infections.
Particularly, we change the parameter $\pi_3$ from $0$ to a large value, say $0.5$,
  and this is a large jump.
Try out the following mod file:
\lstinputlisting[style=Matlab-editor,basicstyle=\mlttfamily\scriptsize,title=\lstname]{progs/replications/Eichenbaum_Rebelo_Trabandt_2022/ert_model_no_homotopy.mod}
Note that even though Dynare has a built in switch to a homotopy method,
  it still fails, because we don't have a large shock or a large difference between initial and terminal conditions,
  but change parameters that affects the dynamics of the model to an endogenous terminal steady-state.
Therefore, we need to implement the homotopy ourselves by looping over the parameter $\pi_3$.
Try out the following mod file with homotopy:
\lstinputlisting[style=Matlab-editor,basicstyle=\mlttfamily\scriptsize,title=\lstname]{progs/replications/Eichenbaum_Rebelo_Trabandt_2022/ert_model_homotopy.mod}

\item[8.] This replicates Figures 5 and 6:
\lstinputlisting[style=Matlab-editor,basicstyle=\mlttfamily\scriptsize,title=\lstname]{progs/replications/Eichenbaum_Rebelo_Trabandt_2022/ert_model_figures_demand.mod}

\item[9.] This replicates Figures 7 and 8:
\lstinputlisting[style=Matlab-editor,basicstyle=\mlttfamily\scriptsize,title=\lstname]{progs/replications/Eichenbaum_Rebelo_Trabandt_2022/ert_model_figures_supply.mod}

\item[10.] This replicates Figures 9 and 10:
\lstinputlisting[style=Matlab-editor,basicstyle=\mlttfamily\scriptsize,title=\lstname]{progs/replications/Eichenbaum_Rebelo_Trabandt_2022/ert_model_figures_both.mod}

\end{itemize}

\fi
\newpage
\end{solution}