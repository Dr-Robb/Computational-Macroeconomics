\section[The Algebra of New Keynesian Models]{The Algebra of New Keynesian Models\label{ex:AlgebraNewKeynesianModels}}
Consider a New Keynesian (NK) model with capital and \textcite{Calvo_1983_StaggeredPricesUtilitymaximizing} price frictions.

\paragraph{Households: Utility}
The economy is assumed to be inhabited by a large representative family with a continuum of members.
Consumption and hours worked are identical across family members.
The household's preferences are defined over per capita consumption $c_t$,
  and per capita labor effort $h_t$.
The representative household maximizes present as well as expected future utility
\begin{align}
	\max E_t \sum_{s=0}^{\infty} \beta^{s} \zeta_{t+s} U(c_{t+s}, h_{t+s}) \label{eq:NewKeynesian.UtilityLifetime}
\end{align}
where $E_t$ is the expectation operator conditional on information at time $t$
  and $\zeta_t$ denotes an exogenous shifter to the discount factor $\beta <1$.
Consider the following functional form for the contemporaneous utility function
\begin{align}
U(c_t, h_t) = \frac{c_t^{1-\sigma_c}}{1-\sigma_c} - \chi_h \frac{h_t^{1+\sigma_h}}{1+\sigma_h} \label{eq:NewKeynesian.UtilityMomentary}
\end{align}
We denote the marginal utilities with respect to consumption and labor as:

{\footnotesize
\begin{align}
U^c_t \equiv \frac{\partial U(c_t,h_t)}{\partial c_t},
\quad
U^h_t \equiv \frac{\partial U(c_t,h_t)}{\partial h_t},
\quad
U^{cc}_t \equiv \frac{\partial^2 U(c_t,h_t)}{\partial^2 c_t},
\quad
U^{ch}_t \equiv \frac{\partial^2 U(c_t,h_t)}{\partial c_t \partial h_t},
\quad
U^{hh}_t \equiv \frac{\partial^2 U(c_t,h_t)}{\partial^2 h_t}
\label{eq:NewKeynesian.UtilityMarginalNotation}
\end{align}
}%
Note that the marginal utility of consumption is positive,
  whereas more labor reduces utility.
Moreover, the chosen utility function \eqref{eq:NewKeynesian.UtilityMomentary} is concave.

~\\\noindent\emph{Exercises:}
\begin{enumerate}
\item Show that $1/\sigma_c$ is the intertemporal elasticity of substitution defined as $IES = -\frac{U_t^c}{U_t^{cc} c_t}$.  
\end{enumerate}

\paragraph{Households: Consumption Bundle}
The consumption good is assumed to be a composite made of a continuum of differentiated goods $c_t(j)$
  represented on the interval $j\in [0,1]$ via a constant elasticity of substitution (CES) aggregation technology as in \textcite{Dixit.Stiglitz_1977_MonopolisticCompetitionOptimum}:
\begin{align}
	c_t = \left(\int_0^1 c_t(j)^{\frac{\epsilon-1}{\epsilon}} dj \right)^{\frac{\epsilon}{\epsilon-1}} \label{eq:NewKeynesian.ConsumptionAggregator}
\end{align}
$\epsilon>1$ is the intratemporal elasticity of substitution across different varieties of consumption goods.
The household decides how to allocate its consumption expenditures among the different goods by taking the price $P_t(j)$ of good $c_t(j)$ as given.

~\\\noindent\emph{Exercises:}
\begin{enumerate}[resume]
\item Show that cost minimization of consumption expenditures implies
\begin{align}
c_t(j) &= \left(\frac{P_t(j)}{P_t}\right)^{-\epsilon} c_t \label{eq:NewKeynesian.ConsumptionDemand}
\\
P_t &= \left(\int_0^1 P_t(j)^{1-\epsilon}dj\right)^{\frac{1}{1-\epsilon}} \label{eq:NewKeynesian.AggregatePriceIndex}
\\
P_t c_t &= \int_0^1 c_t(j) P_t(j) dj \label{eq:NewKeynesian.AggregateConsumptionExpenditures}
\end{align}
Interpret these equations.
\end{enumerate} 

\paragraph{Households: Capital Accumulation}
The household owns the (end of period) capital stock $k_t$ which evolves according to
\begin{align}
k_t = (1-\delta)k_{t-1} + \Biggl( 1 - \underbrace{\frac{\phi_i}{2} \left(\frac{i_t}{i_{t-1}} - 1 \right)^2}_{S\left(\frac{i_t}{i_{t-1}}\right)} \Biggr) i_t \label{eq:NewKeynesian.CapitalAccumulation}
\end{align}
$\delta$ is the depreciation rate and $\phi_i \geq 0$.
$i_t$ denotes gross investment and is assumed to be a composite good made with the same aggregation technology as in equation \eqref{eq:NewKeynesian.ConsumptionAggregator}.
Therefore, bundling is isomorphic to equations \eqref{eq:NewKeynesian.ConsumptionDemand} and \eqref{eq:NewKeynesian.AggregateConsumptionExpenditures}:
\begin{align}
i_t(j) &= \left(\frac{P_t(j)}{P_t}\right)^{-\epsilon} i_t \label{eq:NewKeynesian.InvestmentDemand}
\\
P_t i_t &= \int_0^1 i_t(j) P_t(j) dj \label{eq:NewKeynesian.AggregateInvestmentExpenditures}
\end{align}

~\\\noindent\emph{Exercises:}
\begin{enumerate}[resume]
\item What does the introduction of the function $S\left(\frac{i_t}{i_{t-1}}\right)$ imply for the law of motion of capital?
\end{enumerate}

\paragraph{Households: Budget}
Capital is rented to the intermediate firms at a nominal rate of $R^k_{t}$ which the household takes as given when forming optimal plans.
Similarly, in each period the household takes the nominal wage $W_t$ as given and supplies perfectly elastic labor service $h_t$ to the firm sector.
In return she receives nominal labor income $W_t h_t$.
All firms are owned by the household so that nominal profits and dividends from firms in the final good sector, $ Div^{Fin}_t$,
  as well as from each firm $f\in[0,1]$ in the intermediate goods sector, $\int_0^1 {Div}^{Int}_t(f)df$,
  are received by the household.
Lastly, the household purchases a quantity of one-period nominally risk-free bonds $B_t$ at price $P^B_t$.
The bond matures the following period and pays one unit of money at maturity.
Income and wealth are used to finance consumption expenditures.
In total this defines the \emph{nominal} budget constraint of the household
\begin{align}
\int_0^1 P_t(j) c_t(j) dj + \int_0^1 P_t(j) i_t(j) dj + P^B_t B_t \leq B_{t-1} + W_t h_t + R^k_tk_{t-1} + Div^{Fin}_t + \int_0^1 Div^{Int}_t(f) df
\label{eq:NewKeynesian.BudgetNominal}
\end{align}
In what follows, let lower case letters denote real variables, i.e.\
\begin{align*}
b_t=\frac{B_t}{P_t},~~ w_t=\frac{W_t}{P_t},~~ r^k_t = \frac{R^k_t}{P_t},~~ div^{Fin}_t = \frac{Div^{Fin}_t}{P_t},~~ div^{Int}_t = \frac{Div^{Int}_t}{P_t}
\end{align*}

~\\\noindent\emph{Exercises:}
\begin{enumerate}[resume]

\item Explain the economic intuition behind the following relationships for the nominal interest rate $R_t$ and the real interest rate $r_t$:
\begin{align}
P^B_t &= \frac{1}{R_t} \label{eq:NewKeynesian.NominalInterestRate}
\\
R_t &= r_t E_t \Pi_{t+1} \label{eq:NewKeynesian.RealInterestRate}
\end{align}

\item Derive the intratemporal and intertemporal optimality conditions:
\begin{align}
\lambda_t &= \zeta_t c_t^{-\sigma_c} \label{eq:NewKeynesian.MarginalUtility}
\\
w_t &= \chi_h h_t^{\sigma_h} c_t^{\sigma_c}
\label{eq:NewKeynesian.LaborSupply}
\\
\lambda_t &= \beta E_t \left[\lambda_{t+1} r_t\right]
\label{eq:NewKeynesian.EulerBond}
\\
1 &= q_t \left( 1 - \frac{\phi_i}{2} \left(\frac{i_t}{i_{t-1}}-1\right)^2 - \phi_i \left(\frac{i_t}{i_{t-1}}-1\right)\left(\frac{i_t}{i_{t-1}}\right) \right)
  + \beta E_t \frac{\lambda_{t+1}}{\lambda_t} q_{t+1} \phi_i \left(\frac{i_{t+1}}{i_{t}}-1\right)\left(\frac{i_{t+1}}{i_{t}}\right)^2
\label{eq:NewKeynesian.EulerInvestment}
\\
q_t &= \beta E_t \frac{\lambda_{t+1}}{\lambda_t} \left( r^k_{t+1} + q_{t+1}(1-\delta) \right)
\label{eq:NewKeynesian.EulerCapital}
\end{align}
where $\beta^s\lambda_{t+s}$ and $\beta^s \lambda_{t+s} q_{t+s}$ are the discounted Lagrange multipliers
  corresponding to period $t+s$'s budget constraint \eqref{eq:NewKeynesian.BudgetNominal} in real terms
   and capital accumulation equation \eqref{eq:NewKeynesian.CapitalAccumulation}, respectively.
Interpret these equations.

\item Show that $1/\sigma_h$ is the Frisch elasticity of labor.

\end{enumerate}


\paragraph{Households: Transversality And Solvency Conditions}
It is assumed that the household is subject to a transversality condition
\begin{align*}
	\lim_{T \rightarrow \infty} E_t \left\{\Lambda_{t,T} k_T\right\} = 0
\end{align*}
and a solvency constraint that prevents it from engaging in Ponzi-type schemes:
\begin{align*}
	\lim_{T \rightarrow \infty} E_t \left\{\Lambda_{t,T} \frac{B_T}{P_T}\right\} \geq 0
\end{align*}
for all periods $t$, where
\begin{align}
\Lambda_{t,t+s} = \beta^{s} \frac{\lambda_{t+s}}{\lambda_t} \label{eq:NewKeynesian.StochasticDiscountFactor}
\end{align}
denotes the stochastic discount factor.

~\\\noindent\emph{Exercises:}
\begin{enumerate}[resume]

\item Is there debt in this model? In other words, explain why an optimal path implies
\begin{align}
B_t = 0 \label{eq:NewKeynesian.MarketClearing.Bonds}
\end{align}

\item Explain the difference between the solvency constraint
  $\lim_{T \rightarrow \infty} E_t \left\{\Lambda_{t,T} \frac{B_T}{P_T}\right\} \geq 0$
  and the transversality condition
  $\lim_{T \rightarrow \infty} E_t \left\{\Lambda_{t,T} \frac{B_T}{P_T}\right\} = 0$.
  which holds in the optimum allocation.

\item Derive the following expression for the stochastic discount factor:
\begin{align}
\Lambda_{t,t+1+s} = \beta \frac{\lambda_{t+1}}{\lambda_t} \Pi_{t+1}^{-1} \Lambda_{t+1,t+1+s} \label{eq:NewKeynesian.StochasticDiscountFactorRecursive}
\end{align}

\end{enumerate}

\paragraph{Final Good Firm (Retail Sector): Profit Maximization}
The economy is populated by a continuum of firms indexed by $f \in [0,1]$ that produce differentiated goods $y_t(f)$ in monopolistic competition.
The technology for transforming these intermediate goods into the final output good $y_t$ that can be used for consumption and investment
  has the \textcite{Dixit.Stiglitz_1977_MonopolisticCompetitionOptimum} form:
\begin{align}
y_t = \left[\int\limits_0^1 y_t(f)^{\frac{\epsilon-1}{\epsilon}}df\right]^{\frac{\epsilon}{\epsilon-1}} \label{eq:NewKeynesian.Firms.Aggregator}
\end{align}
where $\epsilon>1$ is the same substitution elasticity as in \eqref{eq:NewKeynesian.ConsumptionAggregator}.

~\\\noindent\emph{Exercises:}
\begin{enumerate}[resume]
\item Show that profit maximization in the final goods sector implies:
\begin{align}
y_t(f) &= \left(\frac{P_t(f)}{P_t}\right)^{-\epsilon} y_t \label{eq:NewKeynesian.Firms.Demand}
\\
P_t y_t &= \int_{0}^{1} P_t(f) y_t(f) df \label{eq:NewKeynesian.Firms.ZeroProfit}
\end{align}
Interpret the equation.
What does this imply for real profits $div_t^{Fin}$ in the final goods sector?
\end{enumerate}

\paragraph{Intermediate Goods Firms (Wholesale Sector): Profit Maximization}
Intermediate firm $f\in[0,1]$ uses the following production function to produce their differentiated good
\begin{align}
y_t(f) = a_t (k_{t-1}(f))^\alpha (n_t(f))^{1-\alpha} \label{eq:NewKeynesian.IntermediateFirms.ProductionFunction}
\end{align}
where $a_t$ denotes the common technology level available to all firms.
Firms face perfectly competitive factor markets for renting capital $k_{t-1}(f)$ and hiring labor $n_t(f)$ with $\alpha$ being a productivity parameter.
Nominal profits of firm $f$ are equal to revenues from selling its differentiated good at self-determined price $P_t(f)$
  minus costs from hiring labor at real wage $w_t$ and real renting rate of capital $r^k_t$:
\begin{align}
{Div}^{Int}_t(f) = P_t(f) y_t(f) - P_t w_t n_t(f) - P_t r^k_t k_{t-1}(f) \label{eq:NewKeynesian.Firms.Profits}
\end{align}
The objective of the firm is to choose contingent plans for $P_t(f)$, $n_t(f)$ and $k_{t-1}(f)$
  so as to maximize the present discounted value of nominal dividend payments given by
\begin{align*}
E_t \sum_{s=0}^{\infty} \Lambda_{t,t+s} Div^{Int}_{t+s}(f)
\end{align*}

~\\\noindent\emph{Exercises:}
\begin{enumerate}[resume]
\item Why are future profits discounted by the household's stochastic discount factor $\Lambda_{t,t+s} = \beta^s \lambda_{t+s}/\lambda_{t}$?
\end{enumerate}

\paragraph{Intermediate Goods Firms (Wholesale Sector): Optimal Factor Inputs}

~\\\noindent\emph{Exercises:}
\begin{enumerate}[resume]

\item Derive the optimal capital and labor demand schedules of intermediate good firm $f$:
\begin{align}
r^k_t  &= mc_t(f) \alpha a_t \left( \frac{n_t(f)}{k_{t-1}(f)}\right)^{1-\alpha}
\label{eq:NewKeynesian.IntermediateFirms.CapitalDemand}
\\
w_t  &= mc_t(f) (1-\alpha) a_t \left(\frac{n_t(f)}{k_{t-1}(f)}\right)^{-\alpha}
\label{eq:NewKeynesian.IntermediateFirms.LaborDemand}
\end{align}
where $mc_t(f)$ is the Lagrange multiplier corresponding to constraint \eqref{eq:NewKeynesian.IntermediateFirms.ProductionFunction}.
Interpret the equations and the explain why $mc_t(f)$ is a measure of real marginal costs.

\item Show that all intermediate firms choose the same capital to labor ratio in production
\begin{align}
\left(\frac{k_{t-1}(f)}{n_t(f)}\right) = \left(\frac{w_t}{1-\alpha}\right) \left(\frac{\alpha}{r^k_t}\right) \label{eq:NewKeynesian.IntermediateFirms.CapitalLaborRatio}
\end{align}

\item Show that marginal costs are independent of $f$:
\begin{align}
mc_t \equiv mc_t(f) = \frac{1}{a_t} \left(\frac{w_t}{1-\alpha}\right)^{1-\alpha} \left(\frac{r^k_t}{\alpha}\right)^{\alpha} \label{eq:NewKeynesian.RealMarginalCosts}
\end{align}

\end{enumerate}

\paragraph{Intermediate Goods Firms (Wholesale Sector): Price Setting}

Prices of intermediate goods are determined by nominal contracts as in \textcite{Calvo_1983_StaggeredPricesUtilitymaximizing} and \textcite{Yun_1996_NominalPriceRigidity}.
In each period firm $f$ faces a constant probability $1-\theta, 0\leq \theta \leq 1$,
  of being able to re-optimize the price $P_t(f)$ of its good $y_t(f)$.
The probability is independent of the time it last reset its price.
Formally:
\begin{align}
P_t(f) = \begin{cases}
\widetilde{P}_t(f) & \text{with probability } 1-\theta\\
P_{t-1}(f) & \text{with probability } \theta
\end{cases}
\label{eq:NewKeynesian.CalvoMechanism}
\end{align}
where $\widetilde{P}_t(f)$ is the reset price in period $t$.
Accordingly, when a firm cannot adjust its price for $s$ periods,
  its price in period $t+s$ is given by $\widetilde{P}_t(f)$ and stays there until the firm can re-optimize again.
Hence, the firm's objective in $t$ is to set $\widetilde{P}_t(f)$ to maximize expected profits
  until it can re-optimize the price again in some future period $t+s$.
The probability to be stuck at the same price for $s$ periods is given by $\theta^s$.

\newpage
~\\\noindent\emph{Exercises:}
\begin{enumerate}[resume]

\item Denote the relative reset price as $\widetilde{p}_t := \frac{\widetilde{P}_t(f)}{P_t}$ and 
  show that optimal price setting of intermediate firms must satisfy:
\begin{align}
\widetilde{p}_t \cdot s_{1,t} &= \frac{\epsilon}{\epsilon-1} \cdot s_{2,t} \label{eq:NewKeynesian.IntermediateFirms.PriceSetting}
\end{align}
where
\begin{align*}
s_{1,t} &= E_t\sum_{s=0}^{\infty}\theta^s \Lambda_{t,t+s} \left(\frac{P_{t+s}}{P_t}\right)^{\epsilon} y_{t+s}
\\
s_{2,t} &= E_t \sum_{s=0}^{\infty}\theta^s \Lambda_{t,t+s} \left(\frac{P_{t+s}}{P_t}\right)^{\epsilon+1} y_{t+s} mc_{t+s}
\end{align*}
Why can we drop the $f$ in the definition of $\widetilde{p}_t$?

\item Show that the infinite sums $s_{1,t}$ and $s_{2,t}$ can be written recursively:
\begin{align}
s_{1,t} &= y_t             + \beta \theta E_t \frac{\lambda_{t+1}}{\lambda_t} \Pi_{t+1}^{\epsilon-1}s_{1,t+1}
\label{eq:NewKeynesian.IntermediateFirms.PriceSettingSum1}
\\
s_{2,t} &=  mc_t y_{t}  + \beta \theta E_t \frac{\lambda_{t+1}}{\lambda_t} \Pi_{t+1}^{\epsilon} s_{2,t+1}
\label{eq:NewKeynesian.IntermediateFirms.PriceSettingSum2}
\end{align}

\item Show that according to the Calvo mechanism the law of motion for the optimal reset price $\widetilde{p}_t = \widetilde{P}_t(f) / P_t$ is given by:
\begin{align}
1&=\theta \Pi_{t}^{\epsilon-1}+\left(1-\theta\right) \widetilde{p}_t^{1-\epsilon}
\label{eq:NewKeynesian.ResetPriceLoM}
\end{align}
\end{enumerate}

\paragraph{Aggregation and Market Clearing}
Note that the capital and labor resource constraints imply that aggregated demands by the intermediate firms need to be equal to the supplies provided by the households:
\begin{align}
h_t = \int_0^1 n_t(f) df \label{eq:NewKeynesian.LaborMarketClearing}
\\
k_{t} = \int_0^1 k_t(f) df \label{eq:NewKeynesian.CapitalMarketClearing}
\end{align}

~\\\noindent\emph{Exercises:}
\begin{enumerate}[resume]

\item Show that the aggregated capital to labor ratio becomes:
\begin{align}
\left(\frac{k_{t-1}}{h_t}\right) = \left(\frac{w_t}{1-\alpha}\right) \left(\frac{\alpha}{r^k_t}\right) \label{eq:NewKeynesian.IntermediateFirms.CapitalLaborRatioAggregated}
\end{align}
Explain the difference between $h_t$ and $n_t(f)$ as well as the difference between $k_t$ and $k_t(f)$.

\item Show that aggregate real profits of the intermediate firms are given by
\begin{align}
div^{Int}_t \equiv \int_{0}^{1} div_t(f) df = y_t - w_t h_t - r^k_t k_{t-1}\label{eq:NewKeynesian.IntermediateFirms.AggregateProfits}
\end{align}
Why is $div^{Int}_t>0$ in this model?

\item Show that aggregation and market clearing implies:
\begin{align}
y_t = c_t + i_t \label{eq:NewKeynesian.AggregateDemand}
\end{align}
Interpret the equation.

\item Show that aggregation of equations \eqref{eq:NewKeynesian.Firms.Demand} and \eqref{eq:NewKeynesian.IntermediateFirms.ProductionFunction} implies:
\begin{align}
p_t^* y_t = a_t k_{t-1}^\alpha h_t^{1-\alpha} \label{eq:NewKeynesian.AggregateSupply}
\end{align}
where
\begin{align}
p_t^* &= \int_{0}^1 \left(\frac{P_t(f)}{P_t}\right)^{-\epsilon} df \label{eq:NewKeynesian.PriceDistortionDefinition}
\end{align}
Interpret the equation and explain why $p_t^*$ is called the \emph{price efficiency distortion}.

\item Using the Calvo mechanism, derive the law of motion for the price efficiency distortion:
\begin{align}
p_{t}^*=\left(1-\theta\right) \widetilde{p}_t^{-\epsilon}+\theta \Pi_{t}^{\epsilon} p_{t-1}^* \label{eq:NewKeynesian.PriceDistortionLoM}
\end{align}

\end{enumerate}

\paragraph{Monetary Policy}
The central bank adjusts the nominal interest rate $R_t$ according to an interest rate rule
  in response to deviations of (i) gross inflation $\Pi_t$ from a target $\Pi^*$
  and (ii) output $y_t$ from steady-state output $y$:
\begin{align}
R_t = R \left(\frac{\Pi_t}{\Pi^*}\right)^{\psi_\pi} \left(\frac{y_t}{y}\right)^{\psi_y} e^{\nu_t} \label{eq:NewKeynesian.MonetaryPolicyRule}
\end{align}
where $R$ denotes the nominal interest rate in steady state,
  $\psi_\pi$ the sensitivity parameter to inflation deviations,
  $\psi_y$ the feedback parameter of the output gap
  and $\nu_t$ an exogenous deviation to the rule.

~\\\noindent\emph{Exercises:}
\begin{enumerate}[resume]
\item Why is such a feedback rule often called a Taylor rule?
\end{enumerate}

\paragraph{Exogenous variables and stochastic shocks}
The exogenous preference shifter $\zeta_t$, the level of technology $a_t$
  and the exogenous deviations $\nu_t$ from the monetary rule evolve according to
\begin{align}
\log{\zeta_t} &= \rho_\zeta \log{\zeta_{t-1}} + \varepsilon_{\zeta,t} \label{eq:NewKeynesian.LoM.PreferenceShifter}\\
\log{a_t} &= \rho_a \log{a_{t-1}} + \varepsilon_{a,t} \label{eq:NewKeynesian.LoM.TFP}\\
\nu_t &= \rho_\nu \nu_{t-1} + \varepsilon_{\nu,t} \label{eq:NewKeynesian.LoM.MonPol}
\end{align}
with persistence parameters $\rho_\zeta$, $\rho_a$ and $\rho_\nu$.
The preference shock $\varepsilon_{\zeta,t}$, the productivity shock $\varepsilon_{a,t}$
  and the monetary policy shock $\varepsilon_{\nu,t}$ are iid Gaussian:
\begin{align*}
\begin{pmatrix}
\varepsilon_{\zeta,t}\\\varepsilon_{a,t}\\\varepsilon_{\nu,t}
\end{pmatrix}
\sim N\left(\begin{pmatrix} 0\\0\\0\end{pmatrix}, \begin{pmatrix} \sigma_\zeta^2 & 0& 0\\0 & \sigma_{a}^2& 0\\0 & 0 & \sigma_{\nu}^2\end{pmatrix}\right)
\end{align*}

\paragraph{Readings}
\begin{itemize}
	\item \textcite{Christiano.Trabandt.Walentin_2010_DSGEModelsMonetary}
	\item \textcite[Ch.3]{Gali_2015_MonetaryPolicyInflation}
	\item \textcite[Ch.19]{Heijdra_2017_FoundationsModernMacroeconomics}
	\item \textcite[Ch.7]{Romer_2019_AdvancedMacroeconomics}
	\item \textcite[Ch.8]{Walsh_2017_MonetaryTheoryPolicy}
	\item \textcite[Ch.3]{Woodford_2003_InterestPricesFoundations}
\end{itemize}

\begin{solution}\textbf{Solution to \nameref{ex:AlgebraNewKeynesianModels}}
\ifDisplaySolutions
\begin{enumerate}

\item Inserting the marginal utilities, $U_t^{c} = c_t^{-\sigma_c}$ and $U_t^{cc} = -\sigma_c c_t^{-\sigma_c-1}$, yields:
\begin{align*}
IES = -\frac{c_t^{-\sigma_c}}{ -\sigma_c c_t^{-\sigma_c-1} c_t} = \frac{1}{\sigma_c}
\end{align*}

\item The household minimizes consumption expenditures
\begin{align*}
\int_0^1 P_t(j) c_t(j)dj
\end{align*}
by choosing $c_t(j)$ and taking the aggregation technology \eqref{eq:NewKeynesian.ConsumptionAggregator} into account.
That is, the Lagrangian is given by:
\begin{align*}
\pounds^c = \int_0^1 P_t(j) c_t(j)dj + P_t \left( c_t - \left[\int_0^1 c_t(j)^{\frac{\epsilon-1}{\epsilon}}dj\right]^{\frac{\epsilon}{\epsilon-1}} \right)
\end{align*}
  where $P_t$ denotes the Lagrange multiplier, i.e.\ the cost of an additional unit in the index $c_t$.
Setting the derivative with respect to $c_t(j)$ equal to zero yields:
\begin{align*}
\frac{\partial \pounds^c}{\partial c_t(j)} = P_t(j) - P_t \left(\frac{\epsilon}{\epsilon-1}\right) \underbrace{\left[\int_0^1 c_t(j)^{\frac{\epsilon-1}{\epsilon}}dj\right]^{\frac{\epsilon}{\epsilon-1}-1}}_{c_t^{1/\epsilon}} \left(\frac{\epsilon-1}{\epsilon}\right) \underbrace{c_t(j)^{\frac{\epsilon-1}{\epsilon}-1}}_{c_t(j)^{-1/\epsilon}} = 0
\end{align*}
which can be simplified to get equation \eqref{eq:NewKeynesian.ConsumptionDemand}:
\begin{align*}
c_t(j) = \left(\frac{P_t(j)}{P_t}\right)^{-\epsilon} c_t
\end{align*}
\emph{Interpretation:} This is the demand function for each consumption good $c_t(j)$.
Accordingly, $\epsilon$ is the (constant) demand elasticity.
	
Plugging this expression into the aggregation technology yields equation \eqref{eq:NewKeynesian.AggregatePriceIndex}:
\begin{align*}
&c_t^{\frac{\epsilon-1}{\epsilon}} = \int_0^1 c_t(j)^{\frac{\epsilon-1}{\epsilon}}dj = \int_0^1 \left(\left(\frac{P_t(j)}{P_t}\right)^{-\epsilon} c_t\right)^{\frac{\epsilon-1}{\epsilon}}dj = c_t^{\frac{\epsilon-1}{\epsilon}} P_t^{\epsilon-1} \int_0^1 P_t(j)^{1-\epsilon}dj
\\
\Leftrightarrow &
P_t = \left[\int_0^1 P_t(j)^{1-\epsilon}dj\right]^{\frac{1}{1-\epsilon}}
\Leftrightarrow
1 = \int_0^1 \left(\frac{P_t(j)}{P_t}\right)^{1-\epsilon}dj
\end{align*}
Similar to the aggregation technology \eqref{eq:NewKeynesian.ConsumptionAggregator} for the consumption index,
  the Lagrange multiplier, that we called $P_t$, can be interpreted as the aggregation technology for the different prices $P_t(j)$.
\\\emph{Interpretation:} $P_t$ is the aggregate price index of consumption bundle $c_t$.

Let's have a look at the objective function, i.e.\ total consumption expenditures $\int_0^1 c_t(j) P_t(j) dj$
  and insert the demand function to get \eqref{eq:NewKeynesian.AggregateConsumptionExpenditures}:
\begin{align*}
\int_0^1 c_t(j) P_t(j) dj = \int_0^1 \left(\frac{P_t(j)}{P_t}\right)^{-\epsilon} c_t P_t(j) dj
= P_t c_t \underbrace{\int_0^1 \left(\frac{P_t(j)}{P_t}\right)^{1-\epsilon}  dj}_{\overset{\eqref{eq:NewKeynesian.AggregatePriceIndex}}{=1}} = P_t c_t
\end{align*}
\emph{Interpretation:} Conditional on optimal behavior of households, total consumption expenditures can be rewritten
  as the product of the aggregate price index times the aggregate consumption quantity index.

\item If $\phi_i=0$, then the capital stock can be adjusted instantaneously.
If, however, $\phi_i>0$, the model features a real rigidity in form of a quadratic adjustment costs of investment.
That is, there is a loss of capital in the investment process,
  which adds a cost to the marginal productivity of capital.
In New Keynesian models, the presence of such investment adjustment costs can lead to a delay in the response of investment to changes in economic conditions;
  and thus better capture the behavior of firms and the implications of investment decisions for the overall economy.
Note that we put a cost on the variation in the level of investment $i_t/i_{t-1}$.
In steady-state there are no costs as $S(1)=0$.
Similarly, the first derivative of $S$ with respect to $(i_t/i_{t-1})$ is also zero: $S'(1)=0$,
  but the second derivative is positive: $S''(i_t/i_{t-1}) = \phi_i >0$.

\item Equation \eqref{eq:NewKeynesian.NominalInterestRate} captures that bond prices are inversely related to interest rates.
When the interest rate goes up, the price of bonds falls.
Intuitively, this makes sense because if you are paying less for a fixed nominal return (at par),
  your expected return should be higher.
More specifically to our model, we consider so-called zero-coupon bonds or discount bonds.
These bonds don't pay any interest but derive their value from the difference between the purchase price
  and the par value (or the face value) paid at maturity.
On maturity the bondholder receives the face value of his investment.
So instead of interest payments, you get a large discount on the face value of the bond;
  that is the price is lower than the face value.
In other words, investors profit from the difference between the buying price and the face value, contrary to the usual interest income.
In our model, we consider zero-coupon bonds with a face value of 1.
So suppose that you buy such a bond at a price of 0.8, then although the bond pays no interest,
  your compensation is the difference between the initial price and the face value.
Let $R_t$ denote the \emph{gross yield to maturity} of a zero-coupon bond,
  that is the discount rate that sets the present value of the promised bond payments equal to the current market price of the bond.
So the price of a Zero-Coupon bond is equal to $P^B_t = \frac{1}{R_t}$.
In the numerical example above this would imply $R_t=1.25$.
As there are no other investment opportunities in this model,
  $R_t$ is also equal to the nominal interest rate in the economy.
This relationship is typically used directly in New Keynesian models,
  so that we replace $P^B_t$ with $1/R_t$ in the budget constraint $\eqref{eq:NewKeynesian.BudgetNominal}$.
	
Equation \eqref{eq:NewKeynesian.RealInterestRate} is the so-called Fisherian equation
  which states that the gross real return on a bond, $r_t$,
  is equivalent to the gross nominal interest rate, $R_t$, divided by the expected gross inflation rate, $E_t \Pi_{t+1}$.
Inflation expectations are responsible for the difference between nominal and real interest rates,
  showing that future expectations matter for the economy.

\item First, let's rewrite the budget constraint in real terms (i.e.\ divide by $P_t$)
  and make use of equations \eqref{eq:NewKeynesian.AggregateConsumptionExpenditures}, \eqref{eq:NewKeynesian.AggregateInvestmentExpenditures} and \eqref{eq:NewKeynesian.NominalInterestRate}:
\begin{align}
c_t + i_t + \frac{1}{R_t} \frac{B_t}{P_t} = \frac{B_{t-1}}{P_{t-1}} \frac{P_{t-1}}{P_t} + \frac{W_t}{P_t} l_t + \frac{R^k_t}{P_t} k_{t-1} + \frac{Div^{Fin}_t}{P_t} + \int_0^1 \frac{Div^{Int}_t(f)}{P_t} df \label{eq:NewKeynesian.BudgetReal}
\end{align}
Furthermore, let lower case letters denote real variables, i.e.\
\begin{align*}
b_t=\frac{B_t}{P_t},~~ w_t=\frac{W_t}{P_t},~~ r^k_t = \frac{R^k_t}{P_t},~~ div^{Fin}_t = \frac{Div^{Fin}_t}{P_t},~~ div^{Int}_t = \frac{Div^{Int}_t}{P_t}
\end{align*} 
  and $\Pi_t=P_t/P_{t-1}$ is the CPI gross inflation rate.
The Lagrangian for the household's problem is then given by:
\begin{align*}
&\pounds^{HH} = E_t \sum_{s=0}^{\infty} \beta^s \zeta_{t+s} \left\{	U\left({c}_{t+s},h_{t+s}\right) \right\}
\\
& - \beta^s \lambda_{t+s} \left\{
  c_{t+s}
+ i_{t+s}
+ \frac{b_{t+s}}{{R}_{t+s}}
- \frac{b_{t-1+s}}{\Pi_{t+s}}
- w_{t+s} h_{t+s}
- r^k_{t+s} k_{t-1+s}
- div^{Fin}_{t+s}
- \int_0^1 div^{Int}_{t+s}(f) df
\right\}
\\
& + \beta^s \lambda_{t+s} q_{t+s} \left\{ (1-\delta) k_{t+s-1} + \Biggl( 1 - \frac{\phi_i}{2} \left(\frac{i_{t+s}}{i_{t-1+s}} - 1 \right)^2 \Biggr) i_{t+s} - k_{t+s}  \right\}
\end{align*}
where $\beta^s\lambda_{t+s}$ and $\beta^s q_{t+s}$ are the scaled Lagrange multipliers
  corresponding to period $t+s$'s \textbf{real} budget constraint \eqref{eq:NewKeynesian.BudgetReal} and capital accumulation equation \eqref{eq:NewKeynesian.CapitalAccumulation}.

\paragraph{First-order condition with respect to $c_t$}
Setting the derivative of $\pounds^{HH}$ with respect to $c_t$ to zero and rearranging directly yields equation \eqref{eq:NewKeynesian.MarginalUtility}:
\begin{align*}
	\lambda_t = \zeta_t U^c_t = \zeta_t c_t^{-\sigma_c}
\end{align*}
\emph{Interpretation:} This is the marginal consumption utility function,
i.e. the benefit (shadow price) of an additional unit of revenue (e.g. dividends, capital or labor income) in the budget constraint.

\paragraph{First-order condition with respect to $h_t$}
Setting the derivative of $\pounds^{HH}$ with respect to $h_t$ to zero and rearranging directly yields equation \eqref{eq:NewKeynesian.LaborSupply}:
\begin{align*}
w_t = - \frac{\zeta_t {U^h_t}}{\lambda_t} = - \frac{U^h_t}{U^c_t} = \chi_h h_t^{\sigma_h} c_t^{\sigma_c}
\end{align*}
\emph{Interpretation:} This is the \textbf{intratemporal} optimality condition or, in other words, the labor supply curve of the household.
Note that the preference shifter $\zeta_t$ has no effect on this intratemporal decision.

\paragraph{First-order condition with respect to $b_t$}
Setting the derivative of $\pounds^{HH}$ with respect to $b_t$ to zero yields:
\begin{align*}
\frac{\lambda_t}{R_t} &= \beta E_t \left[\lambda_{t+1} \Pi_{t+1}^{-1} \right]
\end{align*}
Combine with \eqref{eq:NewKeynesian.RealInterestRate} yields equation \eqref{eq:NewKeynesian.EulerBond}:
\begin{align*}
\lambda_t &= \beta E_t \left[\lambda_{t+1} r_t\right]
\end{align*}
\emph{Interpretation:} This is the so-called Bond Euler equation,
 referring to the optimal \textbf{intertemporal} choice between consumption and saving into bonds.
We have an indifference condition; that is 
  an additional unit of consumption yields either marginal utility today in the amount of $\lambda_t$ (left-hand side);
or, alternatively, this unit of consumption can be saved given the real interest rate $r_t$.
This saved consumption unit has a present marginal utility value of $\beta E_t \lambda_{t+1} r_t$ (right-hand side).
An optimal allocation equates these two choices.

\paragraph{First-order condition with respect to $i_t$}
Setting the derivative of $\pounds^{HH}$ with respect to $i_t$ to zero yields directly equation \eqref{eq:NewKeynesian.EulerInvestment}
\begin{align*}
1 = q_t \left( 1 - \frac{\phi_i}{2} \left(\frac{i_t}{i_{t-1}-1}\right)^2 - \phi_i \left(\frac{i_t}{i_{t-1}}-1\right)\left(\frac{i_t}{i_{t-1}}\right) \right)
+ \beta E_t \frac{\lambda_{t+1}}{\lambda_t} q_{t+1} \phi_i \left(\frac{i_{t+1}}{i_{t}}-1\right)\left(\frac{i_{t+1}}{i_{t}}\right)^2
\end{align*}
\emph{Interpretation:}
This is the Euler equation for investment,
  i.e.\ the optimal choice for choosing investment in the face of adjustment costs.
Note that $\lambda_t q_t$ is the Lagrange multiplier associated with the capital stock
  and thus represents a shadow price of capital;
  this is often referred to as \emph{Tobin's Q},
  which is defined as the ratio of the market value of an asset (like capital) over the replacement cost of that asset.
Accordingly, $q_t$ is the marginal Tobin's Q ratio,
  that measures the additional market value of capital that the households can create by investing in new capital.
If $q_t>1$, then the households value the additional capital more than its costs in consumption terms;
  if $q_t<1$, then the household will delay investment as the benefit is lower than its cost.
Without investment adjustment costs ($\phi_i=0$), the ratio would be 1;
  investment in capital happens instantaneously, there is no wedge between costs and benefits.

\paragraph{First-order condition with respect to $k_t$}
Setting the derivative of $\pounds^{HH}$ with respect to $k_t$ to zero yields directly equation \eqref{eq:NewKeynesian.EulerCapital}
\begin{align*}
q_t &= \beta E_t \frac{\lambda_{t+1}}{\lambda_t} \left( r^k_{t+1} + q_{t+1}(1-\delta) \right)
\end{align*}
\emph{Interpretation:} This is the capital Euler equation.
It is similar to the Bond Euler equation; however, in terms of saving into the capital stock instead of bonds.
Note that due to the same pricing kernel $\Lambda_{t,t+1} = \beta E_t \lambda_{t+1}/\lambda_t$ (aka stochastic discount factor \eqref{eq:NewKeynesian.StochasticDiscountFactor})
  there is arbitrage between the real rate on Bonds and the compensation of capital:
\begin{align*}
r_t = E_t r^k_{t+1} + E_t q_{t+1}(1-\delta)
\end{align*}

\item The Frisch elasticity of labor (FEL) is a measure of how responsive the labor supply is to changes in the real wage rate:
\begin{align*}
FEL = \frac{\frac{\partial{h_t}}{h_t}}{\frac{\partial{w_t}}{w_t}} = \frac{\partial h_t}{\partial w_t} \frac{w_t}{h_t}
\end{align*}
We can compute it by taking the total differential wrt $w_t$ and $h_t$ on equation \eqref{eq:NewKeynesian.LaborSupply}:
\begin{align*}
\partial{w_t} = \chi_h \sigma_h h_t^{\sigma_h-1} c_t^{\sigma_c} \partial{h_t}
\Leftrightarrow
\frac{\partial w_t}{\partial h_t} = \chi_h \sigma_h h_t^{\sigma_h-1} c_t^{\sigma_c}
\end{align*}
Multiplying both sides with $h_t/w_t$ and taking into account equation \eqref{eq:NewKeynesian.LaborSupply} yields:
\begin{align*}
\frac{\partial w_t}{\partial h_t} = \sigma_h = \frac{1}{FEL}
\end{align*}
A high Frisch elasticity of labor means that households are highly responsive to changes in their wage rate and may adjust their work hours or participation in the labor market accordingly.

\item In equilibrium, bond-holding is always zero in all periods: $B_t=0$.
This is due to the fact that in this model we have a representative agent and only private bonds.
If all agents were borrowing, there would be nobody they could be borrowing from.
If all were lenders, nobody would like to borrow from them.
In sum the price of bonds (or more specifically the nominal interest rate) adjusts such
  that bonds across all agents are in zero net supply as markets need to clear in equilibrium.
Note, though, that this bond market clearing condition is imposed \emph{after} you derive the households optimality conditions
  as household savings behavior in equilibrium still needs to be consistent with the bond market clearing.

\item The \emph{No-Ponzi-Game} or \emph{solvency} condition is an external constraint imposed on the individual by the market or other participants.
It is important, because we have an infinite time planing horizon,
  so we forbid agents from acquiring infinite debt that is never repaid, a so-called Ponzi-scheme.
That is, the individual is restricted from financing consumption
  by raising debt and then raising debt again to repay the previous debt
  and finance again consumption and so on.
The individual would very much like to violate it though due to the infinite time horizon,
  so we need to impose this constraint externally.
\textbf{In short:} the \emph{solvency} condition prevents that households consume more than they earn
  and refinance their additional consumption with excessive borrowing.

The \emph{transversality condition} is an optimality condition that states
  that it is not optimal to start accumulating assets and never consume them,
  i.e. $\lim_{T \rightarrow \infty} E_t \left\{\Lambda_{t,T} \frac{B_T}{P_T}\right\} \leq 0$.
But with respect to optimality you would still want to run a Ponzi-scheme if allowed one.
  $\lim_{T \rightarrow \infty} E_t \left\{\Lambda_{t,T} \frac{B_T}{P_T}\right\} \leq 0$
  combined with
  $\lim_{T \rightarrow \infty} E_t \left\{\Lambda_{t,T} \frac{B_T}{P_T}\right\} \geq 0$
  yields
  $\lim_{T \rightarrow \infty} E_t \left\{\Lambda_{t,T} \frac{B_T}{P_T}\right\} = 0$.
This condition must be satisfied in order for the individual to maximize intertemporal utility implying
  that at the limit wealth should be zero.
In other words, if at the limit wealth is positive it means that the household could have increased its consumption without necessarily needing to work more hours;
  thus implying that consumption was not maximized and therefore contradicting the fact that the household behaves optimally.
\textbf{In short:} transversality conditions make sure that households do no have any leftover savings (in terms of bonds or capital)
  as this does not correspond to an optimal path of utility-enhancing consumption.
	
In our model, both the \emph{solvency} and \emph{transversality} conditions for bonds are full-filled already
  as bond-holding is always zero in all periods including the hypothetical asymptotic end of life: $B_t=0$ for all $t$.
So these conditions are rather trivial in this model setting, but are important to impose in more sophisticated models with e.g.\ nonzero government debt.

\item As the firms are owned by the households, the nominal stochastic discount factor,
$\Lambda_{t,t+s}$, between $t$ and $t+s$ is derived from the Euler equation \eqref{eq:NewKeynesian.EulerBond} of the households
$\lambda_t = \beta E_t \left[\lambda_{t+1} R_t \Pi_{t+1}^{-1} \right]$
which implies for the stochastic discount factor \eqref{eq:NewKeynesian.StochasticDiscountFactor}:
\begin{align*}
E_t \Lambda_{t,t+s} = E_t 1/R_{t+s} = E_t\beta^s \frac{\lambda_{t+s}}{\lambda_{t}}\frac{P_t}{P_{t+s}}
\end{align*}
From here, we can establish the following relationships:
\begin{align*}
\Lambda_{t,t} &= 1
\\
\Lambda_{t+1,t+1+s} & = \beta^s \frac{\lambda_{t+1+s}}{\lambda_{t+1}} \frac{P_{t+1}}{P_{t+1+s}}
\\
\Lambda_{t,t+1+s} & = \beta^{s+1} \frac{\lambda_{t+1+s}}{\lambda_{t}} \frac{P_{t}}{P_{t+1+s}} = \beta \frac{\lambda_{t+1}}{\lambda_{t}} \frac{P_{t}}{P_{t+1}} \beta^s \frac{\lambda_{t+1+s}}{\lambda_{t+1}} \frac{P_{t+1}}{P_{t+1+s}} = \beta \frac{\lambda_{t+1}}{\lambda_t} \Pi_{t+1}^{-1} \Lambda_{t+1,t+1+s}
\end{align*}
We will need this later to derive the recursive nonlinear price setting equations.

\item The output packers maximize nominal profits
\begin{align*}
Div_t^{Fin} = P_t div_t^{Fin} = P_t y_t - \int_{0}^{1} P_t(f) y_t(f) df
\end{align*}
subject to \eqref{eq:NewKeynesian.Firms.Aggregator}.
The Lagrangian is
\begin{align*}
\pounds^{p} = P_t y_t - \int_{0}^{1} P_t(f) y_t(f) df + \Lambda_t^p \left\{\left[\int\limits_0^1 y_t(f)^{\frac{\epsilon-1}{\epsilon}}df\right]^{\frac{\epsilon}{\epsilon-1}} - y_t\right\}
\end{align*}
  where $\Lambda_t^p$ is the Lagrange multiplier corresponding to the aggregation technology \eqref{eq:NewKeynesian.Firms.Aggregator}.
The first-order condition w.r.t $y_t$ is
\begin{align*}
P_t = \Lambda_{t}^p
\end{align*}
$\Lambda_{t}^p$ is thus the gain of an additional output unit; hence, equal to the aggregate price index $P_t$.

The first-order condition w.r.t. $y_t(f)$ yields:
\begin{align*}
\frac{\partial \pounds^{p} }{\partial y_t(f)} & = -P_t(f) + \Lambda_{t}^p \frac{\epsilon}{\epsilon-1} \underbrace{\left[\int_{0}^1 y_t(f)^{\frac{\epsilon-1}{\epsilon}}df\right]^{\frac{\epsilon}{\epsilon-1}-1}}_{y_t^{1/\epsilon}} \frac{\epsilon-1}{\epsilon} \underbrace{y_t(f)^{\frac{\epsilon-1}{\epsilon}-1}}_{y_t(f)^{-1/\epsilon}} = 0
\end{align*}
Reordering yields equation \eqref{eq:NewKeynesian.Firms.Demand}
\begin{align*}
y_t(f) = \left(\frac{P_t(f)}{P_t}\right)^{-\epsilon} y_t
\end{align*}
\emph{Interpretation:}
This is the demand curve for intermediate good $y_t(f)$ with constant demand elasticity $\epsilon$.

The aggregate price index is implicitly determined by inserting the demand curve \eqref{eq:NewKeynesian.Firms.Demand} into the aggregator \eqref{eq:NewKeynesian.Firms.Aggregator}:
\begin{align*}
y_t &= \left[\int\limits_0^1 \left(\left(\frac{P_t(f)}{P_t}\right)^{-\epsilon} y_t\right)^{\frac{\epsilon-1}{\epsilon}}df\right]^{\frac{\epsilon}{\epsilon-1}}
\\
\Leftrightarrow
P_t &= \left[\int_{0}^{1} P_t(f)^{1-\epsilon}df\right]^{\frac{1}{1-\epsilon}}
\end{align*}
This is the same price index as in \eqref{eq:NewKeynesian.AggregatePriceIndex}.

Isomorphic to \eqref{eq:NewKeynesian.AggregateConsumptionExpenditures},
  we have that $P_t y_t= \int_{0}^{1} P_t(f) y_t(f) df$ such that dividends of the final goods firm are nil:
\begin{align*}
Div_t^{Fin} = P_t y_t - \int_{0}^{1} P_t(f) y_t(f) df = 0
\end{align*}
This is a manifestation of perfect competition in the retail sector of the economy.

\item As firms are owned by the household, future profits directly increase the revenue side of the households budget constraint \eqref{eq:NewKeynesian.BudgetNominal}.
Additional revenues are valued by their increase in consumption possibilities $\lambda_t$ in period $t$, $\beta \lambda_{t+1}$ in $t+1$ etc.
Therefore we use the stochastic discount factor to compare the benefit of changes in future profits relative to today.

\item The Lagrangian of the intermediate firm is 
\begin{multline}
\pounds^f = E_t \sum_{s=0}^{\infty}\Lambda_{t,t+s} P_{t+s} \left[\frac{P_{t+s}(f)}{P_{t+s}} y_{t+s}(f) - w_{t+s} n_{t+s}(f) - r^k_{t+s} k_{t-1+s}(f) \right.
\\
\left.
+ mc_{t+s}(f)\left(a_{t+s}(k_{t-1+s}(f))^\alpha (n_{t+s}(f))^{1-\alpha} - y_{t+s}(f)\right)\right]
\label{eq:IntermediateFirms.Lagrangian}
\end{multline}
\emph{Interpretation of $mc_t(f)$:} The Lagrange multiplier $mc_t(f)$ is the shadow price of producing an additional output unit in the optimum;
  obviously, this is the definition of real marginal costs.

Taking the derivative wrt $n_t(f)$ and $k_{t-1}(f)$ actually boils down to a static problem (as we only need to evaluate the Lagrangian for $s=0$)
  and yields equations \eqref{eq:NewKeynesian.IntermediateFirms.CapitalDemand} and \eqref{eq:NewKeynesian.IntermediateFirms.LaborDemand}:
\begin{align*}
\frac{r^k_t}{mc_t(f)} &= \alpha a_t \left( \frac{n_t(f)}{k_{t-1}(f)}\right)^{1-\alpha}
\\
\frac{w_t}{mc_t(f)} &= (1-\alpha) a_t \left(\frac{n_t(f)}{k_{t-1}(f)}\right)^{-\alpha}
\end{align*}
\emph{Interpretation:} These are the capital demand and labor demand functions.

\item Dividing equation \eqref{eq:NewKeynesian.IntermediateFirms.CapitalDemand} by \eqref{eq:NewKeynesian.IntermediateFirms.LaborDemand} yields equation \eqref{eq:NewKeynesian.IntermediateFirms.CapitalLaborRatio}:
\begin{align*}
\frac{k_{t-1}(f)}{n_t(f)} = \frac{\alpha w_t}{(1-\alpha)r^k_t}
\end{align*}
\emph{Interpretation:} All firms use the same capital to labor ratio in production as the right-hand-side is independent of $f$.

\item Inserting equation \eqref{eq:NewKeynesian.IntermediateFirms.CapitalLaborRatio} into
  either the capital demand function \eqref{eq:NewKeynesian.IntermediateFirms.CapitalDemand}
  or the labor demand function \eqref{eq:NewKeynesian.IntermediateFirms.LaborDemand}
  yields after some rearranging equation \eqref{eq:NewKeynesian.RealMarginalCosts}:
\begin{align*}
mc_t(f) = \frac{1}{a_t} \left(\frac{w_t}{1-\alpha}\right)^{1-\alpha} \left(\frac{r^k_t}{\alpha}\right)^{\alpha}
\end{align*}
\emph{Interpretation:} As labor and capital inputs are supplied by homogenous factor markets,
  the right-hand side is independent of $f$.
In other words, all firms face the same marginal costs and we denote this by dropping the $f$: $mc_t \equiv mc_t(f)$.
	

\item Let's revisit the Lagrangian of the intermediate firm \eqref{eq:IntermediateFirms.Lagrangian}
  and take the derivative with respect to $P_t(f)$.
To this end, we first focus only on relevant parts in the function and
  substitute the demand function \eqref{eq:NewKeynesian.Firms.Demand} for $y_t(f)$:
\begin{align}
\pounds^{f^p} &= E_t \sum_{s=0}^{\infty}\Lambda_{t,t+s} P_{t+s} \left[
  \left(\frac{P_{t+s}(f)}{P_{t+s}}\right)^{1-\epsilon} y_{t+s}
  - mc_{t+s} \left(\frac{P_{t+s}(f)}{P_{t+s}}\right)^{-\epsilon} y_{t+s}
  + \dots
  \right]
\label{eq:Firms.Lagrangian}
\end{align}
When firms decide how to set their price they need to take into account
that due to the Calvo mechanism they might get stuck at $\widetilde{P}_t(f)$ for a number of periods $s=1,2,...$
before they can re-optimize again.
The probability of such a situation is $\theta^s$.
Therefore, when firms are able to change prices in period $t$,
they take this into account and the above Lagrangian of the expected discounted sum of nominal profits becomes:
\begin{align*}
\pounds^{f^p} &= E_t \sum_{s=0}^{\infty}\theta^s \Lambda_{t,t+s} P_{t+s}\left[ \left(\frac{\widetilde{P}_{t}(f)}{P_{t+s}}\right)^{1-\epsilon} y_{t+s} - mc_{t+s} \left(\frac{\widetilde{P}_{t}(f)}{P_{t+s}}\right)^{-\epsilon} y_{t+s} + \dots \right]\\
&= E_t \sum_{s=0}^{\infty}\theta^s \Lambda_{t,t+s} P_{t+s}^\epsilon y_{t+s} \left[ \widetilde{P}_{t}(f)^{1-\epsilon}  - P_{t+s} \cdot mc_{t+s} \cdot \widetilde{P}_{t}(f)^{-\epsilon} + ... \right]
\end{align*}
The first-order condition of $\pounds^{f^p}$ wrt to $\widetilde{P}_t(f)$ is
\begin{align*}
0= E_t \sum_{s=0}^{\infty}\theta^s \Lambda_{t,t+s} P_{t+s}^\epsilon y_{t+s} \left[ (1-\epsilon)\cdot \widetilde{P}_{t}(f)^{-\epsilon}  +\epsilon \cdot P_{t+s} \cdot mc_{t+s} \widetilde{P}_{t}(f)^{-\epsilon-1}\right]
\end{align*}
As $\widetilde{P}_t(f)>0$ does not depend on $s$, we multiply by $\widetilde{P}_t(f)^{\epsilon+1}$:
\begin{align*}
0= E_t \sum_{s=0}^{\infty}\theta^s \Lambda_{t,t+s} P_{t+s}^{\epsilon} y_{t+s} \left[ (1-\epsilon)\cdot\widetilde{P}_t(f) +\epsilon \cdot P_{t+s} \cdot mc_{t+s}  \right]
\end{align*}
Rearranging
\begin{align*}
\widetilde{P}_t(f) \cdot E_t \sum_{s=0}^{\infty}\theta^s \Lambda_{t,t+s} P_{t+s}^{\epsilon} y_{t+s}  = \frac{\epsilon}{\epsilon-1} \cdot E_t \sum_{s=0}^{\infty}\theta^s \Lambda_{t,t+s} P_{t+s}^{\epsilon+1} y_{t+s} mc_{t+s}
\end{align*}
Dividing both sides by $\color{red}{P_t^{\epsilon+1}}$
\begin{align*}
\underbrace{\frac{\widetilde{P}_t(f)}{\color{red}{P_t}}}_{\widetilde{p}_t} \cdot  \underbrace{E_t\sum_{s=0}^{\infty}\theta^s \Lambda_{t,t+s} \left(\frac{P_{t+s}^{\epsilon}}{\color{red}{P_t^{\epsilon}}}\right) y_{t+s}}_{s_{1,t}}
=
\frac{\epsilon}{\epsilon-1} \cdot \underbrace{E_t \sum_{s=0}^{\infty}\theta^s \Lambda_{t,t+s} \left(\frac{P_{t+s}^{\epsilon+1}}{\color{red}{P_t^{\epsilon+1}}}\right) y_{t+s} mc_{t+s}}_{s_{2,t}}
\end{align*}
The first-order condition can thus be written compactly: 
\begin{align*}
\widetilde{p}_t = \frac{\epsilon}{\epsilon-1} \cdot \frac{s_{2,t}}{s_{1,t}}
\end{align*}
Note that both $s_{1,t}$ and $s_{2,t}$ are independent of $f$; hence, $\widetilde{p}_t$ is also independent of $f$ and therefore we drop the $f$ in the notation.

\emph{Interpretation:} As all firms face the same factor input prices, they choose the same capital to labor ratio and have the same marginal costs.
Therefore, all firms that are allowed to re-optimize will set the same relative reset price $\widetilde{p}_t$.
  

\item To write the two infinite sums recursively,
  we first repeat the previously derived relationship \eqref{eq:NewKeynesian.StochasticDiscountFactorRecursive} for the stochastic discount factor:
\begin{align*}
\color{blue} \Lambda_{t,t+s+1} = \beta \frac{\lambda_{t+1}}{\lambda_t} \Pi_{t+1}^{-1} \Lambda_{t+1,t+1+s}
\end{align*}
and obviously $\Lambda_{t,t}=1$.
  
The first recursive sum can be written as:
\begin{align*}
s_{1,t} &= 
E_t\sum^{\infty}_{\color{red}{s=0}}\theta^s \Lambda_{t,t+s} \left(\frac{P_{t+s}}{P_t}\right)^{\epsilon} y_{t+s}
\\
&= \theta^0 \Lambda_{t,t} \left(\frac{P_t}{P_t}\right)^\epsilon y_t + E_t\sum^{\infty}_{\color{red}{s=1}}\theta^{\color{red}{s}} \Lambda_{t,t+\color{red}{s}} \left(\frac{P_{t+\color{red}{s}}}{P_t}\right)^{\epsilon} y_{t+\color{red}{s}}
\\
&= y_t + E_t\sum^{\infty}_{\color{red}{s=0}}\theta^{\color{red}{s+1}} \Lambda_{t,t+\color{red}{s+1}} \left(\frac{P_{t+\color{red}{s+1}}}{P_t}\right)^{\epsilon} y_{t+\color{red}{s+1}}
\\
&= y_t + E_t\sum^{\infty}_{s=0}\theta^{s+1} {\color{blue}{\Lambda_{t,t+s+1}}} \left({\color{green}{\frac{P_{t+s+1}}{P_{t+1}}\frac{P_{t+1}}{P_{t}}}}\right)^{\epsilon} y_{t+s+1}
\\
&= y_t + E_t\sum^{\infty}_{s=0}\theta^{s+1} {\color{blue}{\beta \frac{\lambda_{t+1}}{\lambda_t}\Pi_{t+1}^{-1} \Lambda_{t+1,t+1+s}}} \left({\color{green}{\frac{P_{t+s+1}}{P_{t+1}}\Pi_{t+1}}}\right)^{\epsilon} y_{t+s+1}
\\
&= y_t + E_t\sum^{\infty}_{s=0}\theta^{s {\color{red}+1}} { {\color{red}\beta \frac{\lambda_{t+1}}{\lambda_t}\Pi_{t+1}^{\epsilon-1}} \Lambda_{t+1,t+1+s}} \left({\frac{P_{t+s+1}}{P_{t+1}}}\right)^{\epsilon} y_{t+s+1}
\\
&= y_t + {\color{red}\theta\beta E_t\frac{\lambda_{t+1}}{\lambda_t}\Pi_{t+1}^{\epsilon-1}} \underbrace{E_t\sum^{\infty}_{s=0}\theta^{s} \Lambda_{t+1,t+1+s} \left(\frac{P_{t+1+s}}{P_{t+1}}\right)^{\epsilon} y_{t+1+s}}_{=s_{1,t+1}}
\end{align*}
The second recursive sum can be written as
\begin{align*}
s_{2,t} &= 
E_t\sum_{{\color{red}s=0}}^{\infty}\theta^s \Lambda_{t,t+s} \left(\frac{P_{t+s}}{P_t}\right)^{\epsilon+1} y_{t+s} mc_{t+s}
\\
&= \theta^0 \Lambda_{t,t} \left(\frac{P_t}{P_t}\right)^{\epsilon+1} y_t mc_t + E_t\sum^{\infty}_{\color{red}{s=1}}\theta^{\color{red}{s}} \Lambda_{t,t+\color{red}{s}} \left(\frac{P_{t+\color{red}{s}}}{P_t}\right)^{\epsilon+1} y_{t+\color{red}{s}} mc_{t+\color{red}{s}}
\\
&= y_t mc_t + E_t\sum^{\infty}_{\color{red}{s=0}}\theta^{\color{red}{s+1}} \Lambda_{t,t+\color{red}{s+1}} \left(\frac{P_{t+\color{red}{s+1}}}{P_t}\right)^{\epsilon+1} y_{t+\color{red}{s+1}} mc_{t+\color{red}{s+1}}
\\
&= y_t mc_t + E_t\sum^{\infty}_{s=0}\theta^{s+1} {\color{blue}{\Lambda_{t,t+s+1}}} \left({\color{green}{\frac{P_{t+s+1}}{P_{t+1}}\frac{P_{t+1}}{P_{t}}}}\right)^{\epsilon+1} y_{t+s+1} mc_{t+s+1}
\\
&= y_t mc_t + E_t\sum^{\infty}_{s=0}\theta^{s+1} {\color{blue}{\beta \frac{\lambda_{t+1}}{\lambda_t}\Pi_{t+1}^{-1}\Lambda_{t+1,t+1+s}}} \left({\color{green}{\frac{P_{t+s+1}}{P_{t+1}}\Pi_{t+1}}}\right)^{\epsilon+1} y_{t+s+1} mc_{t+s+1}
\\
&= y_t mc_t + E_t\sum^{\infty}_{s=0}\theta^{s{\color{red}+1}} {\color{red}\beta \frac{\lambda_{t+1}}{\lambda_t}\Pi_{t+1}^{\epsilon}} \Lambda_{t+1,t+1+s} \left(\frac{P_{t+s+1}}{P_{t+1}}\right)^{\epsilon+1} y_{t+s+1} mc_{t+s+1}
\\
&= y_t mc_t + 
{\color{red}\theta\beta E_t\frac{\lambda_{t+1}}{\lambda_t}\Pi_{t+1}^{\epsilon}} \underbrace{E_t\sum^{\infty}_{s=0}\theta^{s} \Lambda_{t+1,t+1+s} \left(\frac{P_{t+1+s}}{P_{t+1}}\right)^{\epsilon+1} y_{t+1+s} mc_{t+1+s}}_{=s_{2,t+1}}
\end{align*}
\paragraph{Recursive Pricing Summary:}
\begin{align*}
\widetilde{p}_t &\cdot s_{1,t} = \frac{\epsilon}{\epsilon-1} \cdot s_{2,t}\\
s_{1,t} &= y_t + \theta \beta E_t \frac{\lambda_{t+1}}{\lambda_{t}} \Pi_{t+1}^{\epsilon-1} s_{1,t+1}\\
s_{2,t} &= y_t mc_t + \theta \beta E_t \frac{\lambda_{t+1}}{\lambda_{t}} \Pi_{t+1}^{\epsilon} s_{2,t+1}
\end{align*}
Note that the auxiliary variables for the infinite sums are sometimes scaled differently, they are not unique in their definition.
For instance, let's define $\widetilde{s}_{1,t}= \lambda_t s_{1,t}$ and $\widetilde{s}_{2,t}=\lambda_t s_{2,t}$,
  then we can re-write the recursion equivalently by multiplying through with $\lambda_t$:
\begin{align*}
\widetilde{p}_t &\cdot \widetilde{s}_{1,t} = \frac{\epsilon}{\epsilon-1} \cdot \widetilde{s}_{2,t}\\
\widetilde{s}_{1,t} &= y_t \lambda_t + \theta \beta E_t \Pi_{t+1}^{\epsilon-1} \widetilde{s}_{1,t+1}\\
\widetilde{s}_{2,t} &= y_t mc_t \lambda_t + \theta \beta E_t \Pi_{t+1}^{\epsilon} \widetilde{s}_{2,t+1}
\end{align*}
There are many variants of these equations due to different definitions/scalings of the auxiliary variables.
		
\item The law of motion for $\widetilde{p}_t=\frac{\widetilde{P}_t(f)}{P_t}$ is given by combining the aggregate price index \eqref{eq:NewKeynesian.AggregatePriceIndex} with the Calvo mechanism:
\begin{align*}
1 &= \int_{0}^{1} \left(\frac{P_t(f)}{P_t}\right)^{1-\epsilon}df
= \int_{optimizers} \left(\frac{P_t(f)}{P_t}\right)^{1-\epsilon} df  + \int_{non-optimizers} \left(\frac{P_t(f)}{P_t}\right)^{1-\epsilon} df
\end{align*}
That is, $(1-\theta)$ firms can reset their price to $P_t(f) = \widetilde{P}_t(f)$,
  whereas the remaining $\theta$ firms cannot and are stuck at $P_t(f)=P_{t-1}$.
As optimizing and non-optimizing firms behave symmetrically, we therefore have:
\begin{align*}
1&= (1-\theta) \left(\frac{\widetilde{P}_t(f)}{P_t}\right)^{1-\epsilon} + \theta \int_{0}^1 \left(\frac{P_{t-1}(f)}{P_t}{\color{red}{\frac{P_{t-1}}{P_{t-1}}}}\right)^{1-\epsilon}df\\
1&= (1-\theta) \widetilde{p}_t^{1-\epsilon} + \theta \left(\frac{P_{t-1}}{P_{t}}\right)^{1-\epsilon} \int_{0}^1 \left(\frac{P_{t-1}(f)}{P_{t-1}}\right)^{1-\epsilon}df\\
1&=(1-\theta) \widetilde{p}_t^{1-\epsilon}  + \theta \Pi_t^{1-\epsilon} \underbrace{\int_{0}^{1} \left(\frac{P_{t-1}(f)}{P_{t-1}} \right)^{1-\epsilon}df}_{\overset{\eqref{eq:AggregatePriceIndex}}{=}1}\\
1&=(1-\theta) \widetilde{p}_t^{1-\epsilon}  + \theta \Pi_t^{\epsilon-1}
\end{align*}

\item Aggregating equation \eqref{eq:NewKeynesian.IntermediateFirms.CapitalLaborRatio} over $f\in[0,1]$ yields equation \eqref{eq:NewKeynesian.IntermediateFirms.CapitalLaborRatioAggregated}:
\begin{align*}
\underbrace{\int_0^1 k_{t-1}(f)df}_{\overset{\eqref{eq:NewKeynesian.CapitalMarketClearing}}{=}k_{t-1}} 
= \int_0^1 \left(\frac{w_t}{1-\alpha}\right) \left(\frac{\alpha}{r^k_t}\right) n_t(f) df
= \left(\frac{w_t}{1-\alpha}\right) \left(\frac{\alpha}{r^k_t}\right) \underbrace{\int_0^1 n_{t}(f)df}_{\overset{\eqref{eq:NewKeynesian.LaborMarketClearing}}{=}h_{t}}
\end{align*}
$h_t$ is the aggregated \textbf{labor supply} of the households.
Because we assumed that hours worked are identical across family members, there is no need to distinguish between different households $j\in[0,1]$.
In more general models, we would need to aggregate individual labor supply $h_t(j)$ of household $j$.
$n_t(f)$ is the individual \textbf{labor demand} of intermediate firm $f$;
  accordingly, $h_t = \int_0^1 n_t(f) df$ is a resource constraint stating that total hours worked need to equal total labor demand.
Similar arguments apply to the \textbf{capital supply} of the household $k_t$ and individual \textbf{capital demand} $k_t(f)$ of intermediate firm $f$.

\item Aggregated nominal profits of the intermediate firms are given by:
\begin{align*}
Div^{Int}_t \equiv \int_0^1 Div^{Int}_t(f) df =	
\underbrace{\int_0^1 P_t(f) y_t(f) df}_{\overset{\eqref{eq:NewKeynesian.Firms.ZeroProfit}}{=}P_t y_t}
 - P_t w_t \underbrace{\int_0^1 n_t(f) df}_{\overset{\eqref{eq:NewKeynesian.LaborMarketClearing}}{=}h_t}
 - P_t r^k_t \underbrace{\int_0^1 k_{t-1}(f) df}_{\overset{\eqref{eq:NewKeynesian.CapitalMarketClearing}}{=}k_{t-1}}
\end{align*}
Therefore, real profits of the intermediate firms are:
\begin{align*}
 div^{Int}_t = \frac{Div^{Int}_t}{P_t} = y_t - w_t h_t -r^k_t k_{t-1}
\end{align*}
Due to (i) the introduction of monopolistic competition and (ii) nominal price rigidities,
  intermediate firms charge a price larger than their marginal costs
  and will therefore earn profits such that $div^{Int}_t>0$.
However, these two frictions create an inefficient allocation compared to a model without these frictions (aka RBC model).
	
\item Revisit the budget constraint in real terms \eqref{eq:NewKeynesian.BudgetReal} and impose previously derived relationships:
\begin{align*}
c_t &+ i_t + \underbrace{\frac{b_t}{R_t}}_{\overset{\eqref{eq:NewKeynesian.MarketClearing.Bonds}}{=0}}
=
\underbrace{\frac{b_{t-1}}{\Pi_{t}}}_{\overset{\eqref{eq:NewKeynesian.MarketClearing.Bonds}}{=0}}
+ \underbrace{div^{Fin}_t}_{\overset{\eqref{eq:NewKeynesian.Firms.ZeroProfit}}{=0}}
+ \underbrace{\int_0^1 div^{Int}_t(f) df}_{\overset{\eqref{eq:NewKeynesian.IntermediateFirms.AggregateProfits}}{=} y_t - w_t h_t - r^k_t k_{t-1}}
+ w_t h_t + r^k_t k_{t-1}
\\
\Leftrightarrow
c_t &+ i_t = y_t
\end{align*}
\emph{Interpretation:} This is the aggregate demand equation.
	
\item Let's define $y_t^{sum} = \int_{0}^1 y_t(f) df$.
\emph{On the one hand}, we can compute $y_t^{sum}$ using the the production function \eqref{eq:NewKeynesian.IntermediateFirms.ProductionFunction}:
\begin{align*}
y_t^{sum} = \int_{0}^1 y_t(f) df = a_t {\color{red} \int_0^1 \left(\frac{k_{t-1}(f)}{n_{t}(f)}\right)^{\alpha-1}} k_{t-1}(f) {\color{red}df}
\end{align*}
From \eqref{eq:NewKeynesian.IntermediateFirms.CapitalLaborRatio} we know that the labor to capital ratio is independent of $f$; hence:
\begin{align*}
y_t^{sum}  
&\overset{\eqref{eq:NewKeynesian.IntermediateFirms.CapitalLaborRatio}}{=} a_t {\color{red}\left( \left(\frac{w_t}{1-\alpha}\right) \left(\frac{\alpha}{r^k_t}\right) \right)^{\alpha-1}} \int_0^1 k_{t-1}(f)df
\\
&\overset{\eqref{eq:NewKeynesian.CapitalMarketClearing}}{=} a_t {\color{red}\left( \left(\frac{w_t}{1-\alpha}\right) \left(\frac{\alpha}{r^k_t}\right) \right)^{\alpha-1}} k_{t-1}
\end{align*}
From equation \eqref{eq:NewKeynesian.IntermediateFirms.CapitalLaborRatioAggregated} we have
\begin{align*}
\color{red}{\left( \left(\frac{w_t}{1-\alpha}\right) \left(\frac{\alpha}{r^k_t}\right) \right)^{\alpha-1} } = \left(\frac{k_{t-1}}{h_t}\right)^{\alpha-1}
\end{align*}
Inserting into the above expression for $y_t^{sum}$ yields:
\begin{align}
y_t^{sum} = a_t \left(\frac{k_{t-1}}{h_t}\right)^{\alpha-1} k_{t-1} = a_t k_{t-1}^{\alpha} h_t^{1-\alpha} \label{eq:NewKeynesian.ytsum1}
\end{align}

\emph{On the other hand}, we can compute $y_t^{sum}$ with the demand function \eqref{eq:NewKeynesian.Firms.Demand} for intermediate good $y_t(f)$:
\begin{align}
y_t^{sum} = \int_{0}^1 y_t(f) df \overset{\eqref{eq:NewKeynesian.Firms.Demand}}{=} y_t \underbrace{\int_{0}^1 \left(\frac{P_t(f)}{P_t}\right)^{-\epsilon} df}_{=p_t^*} \label{eq:NewKeynesian.ytsum2}
\end{align}
where we made use of the definition of $p_t^*$ in \eqref{eq:NewKeynesian.PriceDistortionDefinition} to \emph{"get rid of the integral"}.

Finally, equating both expressions \eqref{eq:NewKeynesian.ytsum1} and \eqref{eq:NewKeynesian.ytsum2} for $y_t^{sum}$ yields equation \eqref{eq:NewKeynesian.AggregateSupply}:
\begin{align*}
p_t^* y_t =  a_t k_{t-1}^{\alpha} h_t^{1-\alpha}
\end{align*}
\emph{Interpretation:} This is the aggregate supply equation.
Price frictions, however, imply that resources will not be efficiently allocated
  as prices are too high because not all firms can re-optimize their price in every period.
This inefficiency is measured by $p_t^*\leq 1$; therefore it is called the \emph{price efficiency distortion}.

\item The law of motion for the efficiency distortion $p_t^*$ is given due to the Calvo price mechanism, i.e.:
	\begin{align*}
	p_t^* &= \int_0^1\left(\frac{P_t(f)}{P_t}\right)^{-\epsilon} df\\
	p_t^* &= \int_{optimizers} \left(\frac{P_t(f)}{P_t}\right)^{-\epsilon} df + \int_{non-optimizers}\left(\frac{P_t(f)}{P_t}\right)^{-\epsilon} df\\
	p_t^* & = (1-\theta) \widetilde{p}_t^{-\epsilon} + \theta \int_0^1 \left(\frac{P_{t-1}(f)}{P_t}\right)^{-\epsilon} df\\
	p_t^* & = (1-\theta) \widetilde{p}_t^{-\epsilon} + \theta \int_0^1 \left(\frac{P_{t-1}(f)}{P_t }\frac{P_{t-1}}{P_{t-1}}\right)^{-\epsilon} df\\
	p_t^* & = (1-\theta) \widetilde{p}_t^{-\epsilon} + \theta \left(\frac{P_{t-1}}{P_{t}}\right)^{-\epsilon} \int_0^1 \left(\frac{P_{t-1}(f)}{P_{t-1} }\right)^{-\epsilon} df\\
	p_t^* & = (1-\theta) \widetilde{p}_t^{-\epsilon} + \theta \Pi_t^{\epsilon} \underbrace{\int_0^1 \left(\frac{P_{t-1}(f)}{P_{t-1} }\right)^{-\epsilon} df}_{=p_{t-1}^*}\\
	p_t^* & = (1-\theta) \widetilde{p}_t^{-\epsilon} + \theta \Pi_t^{\epsilon} p_{t-1}^*
	\end{align*}


  \item Such a rule was initially proposed by \textcite{Taylor_1993_DiscretionPolicyRules}
  as a statistical descriptor (estimated with OLS) of the Federal Reserve interest setting behavior in response to output gap and inflation.
In \textcite{Taylor_1993_DiscretionPolicyRules}'s preferred specification is given by $\psi_\pi=1.5$ and $\psi_y=0.5$ for a yearly or $\psi_y=0.5/4=0.125$ for a quarterly calibration.
Typically, the rule is augmented with one lag of the nominal interest rate to address strong persistence in the evolution of the nominal interest rate.
Also, some models feature a forward-looking behavior of the central bank arguing that it responds to forecasts of inflation and output gap.

\end{enumerate}
\fi
\newpage
\end{solution}