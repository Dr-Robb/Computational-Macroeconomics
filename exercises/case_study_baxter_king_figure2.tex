\section[Case Study Baxter and King (1993, AER): Macroeconomic Effects Of A Permanent Increase in Basic Government Purchases]{Case Study Baxter and King (1993, AER): Macroeconomic Effects Of A Permanent Increase in Basic Government Purchases\label{ex:CaseStudy.BaxterKing.Figure2}}

\begin{enumerate}
    \item Read the paper by \textcite{Baxter.King_1993_FiscalPolicyGeneral} and focus particularly on the model equations and analysis done in section III.B.
    \item Write a Dynare mod file for the model and calibrate it according to the \emph{Benchmark Model with Basic Government Purchases} given in Table 1.
    Compute the steady-state using an \texttt{initval} block.
    \item Replicate Figure 2.
\end{enumerate}
\paragraph{Notes and hints}
\begin{itemize}
    \item According to footnote 3, the quantitative analysis corresponds to a model with labor-augmenting technical progress.
    Calibrate $\gamma_X$ such that the economy grows at 1.6\% per annum, a typical value used in e.g.\ \textcite{King.Plosser.Rebelo_1988_ProductionGrowthBusiness}.
    \item Pay attention to the different scales on the y-axes, i.e.\ commodity units, percent, and basis points.
    \item Skip the term structure variable in panel \emph{C.}, as we will re-visit the term structure in RBC models later on.
    \item Use a perfect foresight simulation for computing the transition path
      (i) from the initial steady-state (specified by \texttt{initval}) to
      (ii) a terminal steady-state (specified by \texttt{endval}) with
      (iii) a permanent increase in basic government purchases (specified by setting a nonzero value on the exogenous variable in the terminal steady-state).
    The following code snippet might be useful:
\end{itemize}

{\footnotesize
\begin{lstlisting}[style=Matlab-editor,basicstyle=\mlttfamily\scriptsize]
idx_y  = varlist_indices('y', M_.endo_names); % index to access y in oo_.endo_simul

% STEP 1: SET INITIAL CONDITION
initval;
% TODO: add values for all variables with the initial steady-state
end;
steady; % compute initial steady-state
        % steady after intival uses the computed steady-state to initialize oo_.endo_simul

% STEP 2: SET TERMINAL CONDITION
commodity_unit = 0.01*oo_.steady_state(idx_y); % commodity unit (p.321) defined as 1% of initial output

endval;
% TODO: set a nonzero value for the exogenous variable on basic government spending to implement a permanent increase
% other variables that are not specified in the endval block remain at the initial steady-state
end;

steady; % recompute steady-state numerically starting from old steady-state and changed values
        % steady after endval will set the terminal condition oo_.endo_simul(:,end) to the endval steady-state

% STEP 3: COMPUTE TRANSITION PATH
perfect_foresight_setup(periods=200);
perfect_foresight_solver;

% STEP 4: PLOTTING
% oo_.endo_simul contains the simulated variables,
% note that the first entry is the initial condition (initval steady-state)
% and the last is the terminal condition (endval steady-state)
horizon = 1:22;
% do transformations on plotted variables
plot_y  = (oo_.endo_simul(idx_y,horizon+1)-oo_.endo_simul(idx_y,1))./commodity_unit;   % deviation from initial steady-state, normalized in commodity units
plot(horizon,plot_y, "o",'MarkerEdgeColor','black','MarkerFaceColor','black','MarkerSize',8,'LineStyle','none');

\end{lstlisting}
}%

\begin{solution}\textbf{Solution to \nameref{ex:CaseStudy.BaxterKing.Figure2}}
\ifDisplaySolutions
\lstinputlisting[style=Matlab-editor,basicstyle=\mlttfamily\scriptsize,title=\lstname]{progs/replications/Baxter_King_1993/Baxter_King_1993_figure_2.mod}
\fi
\newpage
\end{solution}